\chapter*{Введение}

\addcontentsline{toc}{chapter}{Введение}

Под векторными задачами мы будем понимать задачи, в которых решением является некоторая вектор-функция. Будем рассматривать такие векторные задачи, решениями которых являются вектор-функции с компонентами, каждая из которых будет удовлетворять дифференциальному уравнению второго порядка и как минимум непрерывна. Таким образом, каждая из компонент искомой вектор-функции может быть найдена в виде линейной комбинации непрерывных базисных функции, которые использовались при решении скалярных задач. Такие базисные функции обычно называют узловыми (к ним относятся не только лагранжевы и эрмитовы базисные функции, но и иерархические). Соответственно и МКЭ, использующий при нахождении численного решения, такие базисные функции называют узловым.

Технологию построения конечноэлементных аппроксимации векторных задач на основе узлового МКЭ мы рассмотри на примере задачи (), описывающие нестационарное электромагнитное поле в однородной по магнитной проницаемости среде (и без учета токов смещения).
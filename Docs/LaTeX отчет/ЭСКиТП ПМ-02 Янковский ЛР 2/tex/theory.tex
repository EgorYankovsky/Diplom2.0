\chapter{Теоретическая часть}

\section{Условие задачи}

Пусть имеется некоторый круглый индукционный источник, с радиусом $R_0$ $<<$ 1000. На рисунке \ref{fig:areaExample} имеем однородные краевые условие на правой и нижней границах, и естественные на левой и верхней границах.

\begin{figure}
	\centering
	\includegraphics[width=0.75\linewidth]{images/"Образец сетки".png}
	\caption{Образец сетки}
	\label{fig:areaExample}
\end{figure}

\section{Математическая постановка}

Будем считать, что электромагнитное поле возбуждается круговым током, а вмещающая среда имеет круговую симметрию. Тогда при условии однородности среды по магнитной проницаемости электромагнитное поле полностью описывается одной компонентой $A_{\varphi} = A_{\varphi}(r, z, t)$ вектор-потенциала $\overline{A}$ (в цилиндрической системе координат), и эта функция $A_{\varphi}(r, z, t)$ может быть найдена из решения двумерного уравнения:

\begin{equation} \label{eq1M}
	-\frac{1}{\mu_0} \Delta A_{\varphi} + \frac{A_{\varphi}}{\mu_0 r^2} + \sigma \frac{\partial A_{\varphi}}{\partial t} = J_{\varphi},
\end{equation}
где: $J_{\varphi}$ - дельта-функция равная 1 в одной из подобластей, описывающей кольцо, и равная 0 в остальных.

Переведем это дифференциальное уравнение в частных производных в слабую форму.

\begin{equation} \label{eq2}
	\int \limits_{\Omega} \left(-\frac{1}{\mu_0} \grad \left(\grad{A_{\varphi}}\right) + \frac{A_{\varphi}}{\mu_0 r^2} + \sigma \frac{\partial A_{\varphi}}{\partial t}\right) v d \Omega = \int \limits_{\Omega} J_{\varphi} v d \Omega.
\end{equation}

\begin{equation} \label{eq3}
	\int \limits_{\Omega} \grad \left( -\frac{1}{\mu_0} \grad{A_{\varphi}}\right) v d \Omega + \int \limits_{\Omega} \frac{A_{\varphi}}{\mu_0 r^2} v d \Omega + \int \limits_{\Omega} \sigma \frac{\partial A_{\varphi}}{\partial t} v d \Omega = \int \limits_{\Omega} J_{\varphi} v d \Omega.
\end{equation}

Применив формулу Гаусса-Остроградского, и принимая во внимание, что по условию задачи в некоторых местах поток через границу равен нулю, получим:

\begin{equation} \label{eq4}
	\int \limits_{\Omega} \frac{1}{\mu_0}   \grad{A_{\varphi}} \grad v d \Omega + \int \limits_{\Omega} \frac{A_{\varphi}}{\mu_0 r^2} v d \Omega + \int \limits_{\Omega} \sigma \frac{\partial A_{\varphi}}{\partial t} v d \Omega - \int \limits_{\Omega} J_{\varphi} v d \Omega = 0.
\end{equation}

\section{Принципы построения локальных векторов, матриц жесткости и масс}
Поскольку решаемое уравнение в $(r, z)$ координатах и имеется особый нелинейный коэффициент $\gamma = \frac{1}{r^2}$, локальные матрицы жесткости и масс для одномерной задачи выглядят следующим образом:

\begin{equation*}
	\hat{G^r} = \hat{\lambda} \frac{r_k + h_k / 2}{h_k} \left(
	\begin{array}{rr}
		 1 & -1\\
		-1 &  1\\
	\end{array}
	\right),
\end{equation*}

\begin{equation*}
    \hat{M^r} = \ln\left(1 + \frac{1}{d}\right)
	\left(
	\begin{array}{cc}
		(1+d)^2 & -d(1+d)\\
		-d(1+d) &  d^2\\
	\end{array}
	\right)
	-d
	\left(
	\begin{array}{rr}
		1 & -1\\
		-1 & 1\\
	\end{array}
	\right)
	+ \frac{1}{2}
	\left(
	\begin{array}{rr}
		-3 & 1\\
		1 & 1\\
	\end{array}
	\right)
\end{equation*}
где $d = \frac{r_k}{h_k}$.


\begin{equation*}
	\hat{G^z} = \frac{\hat{\lambda}}{h_k} \left(
	\begin{array}{rr}
		1 & -1\\
		-1 &  1\\
	\end{array}
	\right),
\end{equation*}

\begin{equation*}
	\hat{M^z} = \frac{\hat{\gamma} h_k}{6} \left(
	\begin{array}{rr}
		2 & 1\\
		1 & 2\\
	\end{array}
	\right).
\end{equation*}

Тогда элементы верхнего треугольника матрицы жесткости для двумерных задач, можем представить в виде:

\begin{equation*}
	\begin{array}{ll}
		\hat{G}_{11} = \hat{\lambda}\left(G^r_{11}M^z_{11} + M^r_{11}G^z_{11}\right), & \hat{G}_{12} = \hat{\lambda}\left(G^r_{12}M^z_{11} + M^r_{12}G^z_{11}\right),\\
		\hat{G}_{13} = \hat{\lambda}\left(G^r_{11}M^z_{12} + M^r_{11}G^z_{12}\right), & \hat{G}_{14} = \hat{\lambda}\left(G^r_{12}M^z_{12} + M^r_{12}G^z_{12}\right),\\
		\hat{G}_{22} = \hat{\lambda}\left(G^r_{22}M^z_{11} + M^r_{22}G^z_{11}\right), & \hat{G}_{23} = \hat{\lambda}\left(G^r_{21}M^z_{12} + M^r_{21}G^z_{12}\right),\\
		\hat{G}_{24} = \hat{\lambda}\left(G^r_{22}M^z_{12} + M^r_{22}G^z_{12}\right), & \hat{G}_{33} = \hat{\lambda}\left(G^r_{11}M^z_{22} + M^r_{11}G^z_{22}\right),\\
		\hat{G}_{34} = \hat{\lambda}\left(G^r_{12}M^z_{22} + M^r_{12}G^z_{22}\right), & \hat{G}_{44} = \hat{\lambda}\left(G^r_{22}M^z_{22} + M^r_{22}G^z_{22}\right).\\
	\end{array}
\end{equation*}

Верхний треугольник элементов матрицы масс может быть представлены в виде:

\begin{equation*}
	\begin{array}{ll}
		\hat{M}_{11} = \hat{\gamma}M^r_{11}M^z_{11}, & \hat{M}_{12} = \hat{\gamma}M^r_{12}M^z_{11},\\
		\hat{M}_{13} = \hat{\gamma}M^r_{11}M^z_{12}, & \hat{M}_{14} = \hat{\gamma}M^r_{12}M^z_{12},\\
		\hat{M}_{22} = \hat{\gamma}M^r_{22}M^z_{11}, & \hat{M}_{23} = \hat{\gamma}M^r_{21}M^z_{12},\\
		\hat{M}_{24} = \hat{\gamma}M^r_{22}M^z_{12}, & \hat{M}_{33} = \hat{\gamma}M^r_{11}M^z_{22},\\
		\hat{M}_{34} = \hat{\gamma}M^r_{12}M^z_{22}, & \hat{M}_{44} = \hat{\gamma}M^r_{22}M^z_{22}.\\
	\end{array}
\end{equation*}

Выразим матрицу $\hat{M}$ следующим образом:

\begin{equation*}
	\hat{M} = \hat{\gamma} \hat{C}.
\end{equation*}

Для генерации вектора правой части, воспользуемся следующим соотношением:

\begin{equation*}
	\hat{b} = \hat{C} \hat{f}.
\end{equation*}
 
 
\section{Аппроксимация краевой задачи по времени}

Представим искомое решение $u$ на интервале $\left(t_{j-2}, t_j\right)$ в следующем виде:

\begin{equation} \label{eq5m}
	u(r, z, t) \approx u^{j-2}(r, z)\eta_2^j(t) + u^{j-1}(r, z)\eta_1^j(t) + u^{j}(r, z)\eta_0^j(t).
\end{equation}

где функции $\eta_2^j(t)$, $\eta_1^j(t)$, $\eta_0^j(t)$ - базисные квадратичные полиномы Лагранжа (с двумя корнями из набора значений времен $t_{j-2}$, $t_{j-1}$, $t_j$), которые могут быть записаны в виде:

\begin{equation*}
	\eta_2^j(t) = \frac{1}{\Delta t_1 \Delta t} \left(t - t_{j-1}\right) \left(t-t_j\right),
\end{equation*}

\begin{equation*}
	\eta_1^j(t) = -\frac{1}{\Delta t_1 \Delta t_0} \left(t - t_{j-2}\right) \left(t-t_j\right),
\end{equation*}

\begin{equation*}
	\eta_0^j(t) = \frac{1}{\Delta t \Delta t_0} \left(t - t_{j-2}\right) \left(t-t_{j-1}\right),
\end{equation*}
где:

\begin{equation*}
	\begin{array}{ccc}
	\Delta t = t_j - t_{j - 2},&
	\Delta t_1 = t_{j - 1} - t_{j - 2},&
	\Delta t_0 = t_j - t_{j - 1}.
	\end{array}
\end{equation*}


Применим представление \ref{eq5m} для аппроксимации производной по времени параболического уравнения \ref{eq1M} на временном слое $t = t_j$:

\begin{equation} \label{cabel}
	\sigma \frac{\partial}{\partial t} \left(u^{j-2}(r, z) \eta_2^j(t) + u^{j-1}(r, z) \eta_1^j(t) + u^j(r, z) \eta_0^j(t)\right) -\frac{1}{\mu_0} \Delta A_{\varphi} + \frac{A_{\varphi}}{\mu_0 r^2} = J_{\varphi}
\end{equation}

Выполняя конечноэлементную аппроксимацию краевой задачи для уравнения \ref{cabel}, получим СЛАУ следующего вида:


\begin{multline} \label{gabel}
	\left(\frac{\Delta t + \Delta t_0}{\Delta t \Delta t_0} M + G + M\right)q^{j} = b^{j} - \frac{\Delta t_0}{\Delta t \Delta t_1} M q^{j - 2} + \frac{\Delta t}{\Delta t_1 \Delta t_0} M q^{j - 1}.
\end{multline}

% Далее глава тестирования.

\chapter{Описание программы}

\section{Описание расчётной области}

Расчётная область является набором прямоугольников в декартовой систе-ме координат. Пример расчётной области представлен на рисунке \ref{fig:exampleOfArea}.
\begin{figure}
    \centering
    \includegraphics[width=0.75\linewidth]{images/1.png}
    \caption{Пример расчётной области}
    \label{fig:exampleOfArea}
\end{figure}

Для описания расчётной области используются структуры данных, расположенные в файлах Settings.dat, Splitting.dat, BoundaryConditions.dat, Time.dat. Вид этих структур представлен в таблицах \ref{tab:setdatstr}-\ref{tab:timedatstr} Более подробно о принципе создания сетки описывается в \cite{1}. Чтобы показать, что на все пункты в списке литературы имеются перекрёстные ссылки, добавим этот фрагмент текста и ссылки \cite{2}, \cite{3}, \cite{4}. Эти источники были добавлены для примера оформления документа.

\begin{table}
    \caption{Структура файла Settings.dat}
    \centering
    \small
    \begin{tabularx}{1.0\textwidth}{| >{\raggedright\arraybackslash}X | >{\raggedright\arraybackslash}X | }
        \hline
        \centering{Данные в файле} & \centering{Пояснение} \tabularnewline
        \hline
\texttt{3 \newline
        -4 2 4 \newline
        3 \newline
        0 1 4 \newline
        2 \newline
        1 1 2 1 3 \newline
        2 2 3 2 3}
        & Количество X – линий (вертикальные) \newline
            X – линии \newline
            Количество Y – линий (горизонтальные) \newline
            Y – линии \newline
            Количество подобластей \newline
            Пять чисел (номер формул подобласти, индекс начала в массиве X – линий, индекс конца в масси-ве X – линий, индекс начала в массиве Y – линий, индекс конца в массиве Y – линий) \tabularnewline
        \hline
    \end{tabularx}
    \label{tab:setdatstr}
\end{table}

\begin{table}
    \caption{Структура файла Splittings.dat}
    \centering
    \small
    \begin{tabularx}{1.0\textwidth}{| >{\raggedright\arraybackslash}X | >{\raggedright\arraybackslash}X | }
        \hline
        \centering{Данные в файле} & \centering{Пояснение} \tabularnewline
        \hline
\texttt{6 1 2 1 \newline
        1 1 2 2}

        & 
        Количество разбиений и коэффициент разрядки (N - 1 раз, где N – количество элементов массива X – линий) \newline
        Количество разбиений и коэффициент разрядки (N - 1 раз, где N – количество элементов массива Y – линий) \tabularnewline
        \hline
    \end{tabularx}
    \label{tab:splitdatstr}
\end{table}

\begin{table}
    \caption{Структура файла BoundaryConditions.dat}
    \centering
    \small
    \begin{tabularx}{1.0\textwidth}{| >{\raggedright\arraybackslash}X | >{\raggedright\arraybackslash}X | }
        \hline
        \centering{Данные в файле} & \centering{Пояснение} \tabularnewline
        \hline
\texttt{2 1 1 1 1 3 \newline 
        1 1 1 2 1 1 \newline
        1 1 2 2 1 2 \newline
        1 1 2 3 2 2 \newline
        1 1 3 3 2 3 \newline
        1 1 1 3 3 3 }
        & 
        Шесть чисел в строке: \newline
        тип краевого условия, \newline
        номер формул краевого условия, \newline
        индекс начала в массиве X – линий, \newline
        индекс конца в массиве X – линий, \newline
        индекс начала в массиве Y – линий, \newline
        индекс конца в массиве Y – линий) \tabularnewline
        \hline
    \end{tabularx}
    \label{tab:bcdatstr}
\end{table}

\begin{table}
    \caption{Структура файла Time.dat}
    \centering
    \small
    \begin{tabularx}{1.0\textwidth}{| >{\raggedright\arraybackslash}X | >{\raggedright\arraybackslash}X | }
        \hline
        \centering{Данные в файле} & \centering{Пояснение} \tabularnewline
        \hline
\texttt{0 5 \newline
        5 \newline
        1}
        & 
        Начало временного отрезка, конец временного отрезка \newline
        Количество разбиений \newline
        Коэффициент разрядки \tabularnewline
        \hline
    \end{tabularx}
    \label{tab:timedatstr}
\end{table}

\section{Структура модулей программы}
Генерация портрета СЛАУ, вычисление локальных матриц и генерация глобальных матриц описываются в классе \texttt{FEM} в файле FEM.cs, метод решения СЛАУ описывается в классе \texttt{Solver} в файле Solver.cs.
\documentclass[14pt,a4paper]{extreport}

\usepackage{style/style}
\usepackage{physics}
\usepackage{fancyhdr}

\fancypagestyle{plain}{%
\fancyhf{} % clear all header and footer fields
\fancyfoot[C]{\small\thepage}}
\renewcommand{\headrulewidth}{0pt}
\renewcommand{\footrulewidth}{0pt}
\pagestyle{plain}

\makeatletter
  \def\my@tag@font{\small}
  \def\maketag@@@#1{\hbox{\m@th\normalfont\my@tag@font#1}}
  \let\amsmath@eqref\eqref
  \renewcommand\eqref[1]{{\let\my@tag@font\relax\amsmath@eqref{#1}}}
\makeatother

\usepackage{titletoc}
\titlecontents{chapter}[0em]{\bfseries}{\thecontentslabel.\hspace{1em}}{}{\titlerule*[1pc]{.}\contentspage}
\titlecontents{section}[1.25em]{}{\thecontentslabel.\hspace{1em}}{}{\titlerule*[1pc]{.}\contentspage}
\titlecontents{subsection}[2.5em]{}{\thecontentslabel.\hspace{1em}}{}{\titlerule*[1pc]{.}\contentspage}

\begin{document}

% Отключение нумерации страниц
\pagenumbering{gobble}

%\chapter*{Аннотация}

Отчёт 100500 с., 4 ч., 100501 рис., 100502 табл., 100503 источника, 100504 прил.

МОДЕЛИРОВАНИЕ ЭЛЕКТРОМАГНИТНОГО ПОЛЯ, МЕТОД КОНЕЧНЫХ ЭЛЕМЕНТОВ, МНОГОЭТАПНАЯ СХЕМА РАЗДЕЛЕНИЯ ПОЛЕЙ

\textbf{Объект исследования:}  математическая модель электромагнитного поля в земной полости, создаваемое индукционным источником.
 
\textbf{Цель работы:} 
Сравнение результатов при использовании разделения поля на первичное и вторичное с результатами многоэтапной схемы разделения полей.

\textbf{Гипотеза исследования:} планомерное рассеивание поля в верних слоях земной коры. Сокращение вычислительных затрат при реализации многоэтапного разделения части поля.

\newpage

\tableofcontents
\newpage

% Включение нумерации страниц
\pagenumbering{arabic}
\setcounter{page}{3}
\chapter*{Введение}

\addcontentsline{toc}{chapter}{Введение}

С увеличением потребности в природных ресурсах развивались способы поиска и исследования земных пород и руд. Наиболее распространёнными являются гравиразведка, магниторазведка, и электроразведка. Первые два можно считать естественными или природными, поскольку в их основе лежит использование гравитационного и магнитного полей Земли. При использовании электроразведки, поле создаётся искусственными источниками. В качестве данных источников можно рассматривать петлевые источники, вертикальные электрические линии (ВЭЛ) и самый новый из предложенных -- круговой электрический диполь (КЭД). Новизна его заключается в том, что это источник переменного поля, наземный аналог вертикальной электрической линии. 

Помимо возможности нахождения параметров среды, что является обратной задачей по определению, можно изучать и поведение самого электромагнитного поля. Зная его источник и параметры среды можно воспроизвести поле в пространстве земной коры, изучать характер его поведения в зависимости от количества неоднородностей в земной среде или параметров горизонтально-слоистой среды Земли.

При моделирования таких процессов, а особенно при истолковании сложных полей в которых было необходимо преобразовывать таким образом, чтобы разделить аномалии в зависимости от глубины расположения источников поля и обособлять такие изменения поля, которые соответствуют аномалиям тел простейших форм, обычные аналитические методы математического моделирования не помогут в силу больших временных затрат на расчёты. Вместо этого пользуются различными численными методами на различных ЭВМ. Для моделирования электромагнитных полей, лучшим вариантом можно считать метод конечных элементов (МКЭ). 

В данной работе будет рассмотренна возможность расчёта нестационарного поля КЭД в осесимметричной среде. Исследования проводились на воображаемой области размером $\Omega \in [0.001; 1000]_r \times [-1000; 0]_z$. Данная область включает в себя несколько разных слоев земных пород, имеющих различные удельные значения физических величин. Также рассмотрим процесс добавления нескольких аномальных пород, характерных различным земным рудам.

При написании работы использовались следующие языки программирования: для математических расчётов -- C$\#$ 12 на платформе .NET 8.0, для визуализации полученных результатов -- Python 3.12.2 с пакетами matplotlib версии 3.8.2 и numpy версии 1.26.4. 

%\chapter{Исследования}

\section{Исследование первичного поля}

Пусть источник индукционного поля лежит на расстоянии $R = 500$ м от оси симметрии и имееет силу тока, равную $J_{\varphi} = 1.0$ А. Также условимся, что источник работал достаточно долго, чтобы создать стабильное электромагнитное поле. Сетка по времени равномерная: $t=[1.0; 1.05]$ на 200 временных слоёв. После истечения первой секунды мы отключим наш источник, т.е. $J_{\varphi} = 0.0$ A при $t > 1.0$. На рисунках \ref{fig:A_phi_0} -- \ref{fig:E_phi_2} представлено распространение этого поля в среде в начальный, промежуточных и последний момент времени.

\begin{figure}
	\centering
	\includegraphics[width=1.0\linewidth]{images/Answer_A_time_layer_1.png}
	\caption{Решение $A_{\varphi}$ при $t = 1.0с$}
	\label{fig:A_phi_0}
\end{figure}

\begin{figure}
	\centering
	\includegraphics[width=1.0\linewidth]{images/Answer_E_time_layer_1.png}
	\caption{Решение $E_{\varphi}$ при $t = 1.0с$}
	\label{fig:E_phi_0}
\end{figure}

\begin{figure}
	\centering
	\includegraphics[width=1.0\linewidth]{images/Answer_A_time_layer_1.0250000000000083.png}
	\caption{Решение $A_{\varphi}$ при $t = 1.025с$}
	\label{fig:A_phi_1}
\end{figure}

\begin{figure}
	\centering
	\includegraphics[width=1.0\linewidth]{images/Answer_E_time_layer_1.0250000000000083.png}
	\caption{Решение $E_{\varphi}$ при $t = 1.025с$}
	\label{fig:E_phi_1}
\end{figure} 

\begin{figure}
	\centering
	\includegraphics[width=1.0\linewidth]{images/Answer_A_time_layer_1.05.png}
	\caption{Решение $A_{\varphi}$ при $t = 1.05с$}
	\label{fig:A_phi_2}
\end{figure}

\begin{figure}
	\centering
	\includegraphics[width=1.0\linewidth]{images/Answer_E_time_layer_1.05.png}
	\caption{Решение $A_{\varphi}$ при $t = 1.05с$}
	\label{fig:E_phi_2}
\end{figure} 

Расположим на расчётной области приёмники в каждой горизонтально-слоистой среде и проведем замеры значений вектор-потенциала и электрического поля в точках $(2500; 0; -100)$, $(2500; 0; -200)$, $(10; 0; -700)$, $(1000; 0; -1250)$. Отобразим на графиках \ref{fig:NatA} -- \ref{fig:LogE} полученные значения.

\begin{figure}
	\centering
	\includegraphics[width=0.8\linewidth]{images/Normal_A.png}
	\caption{Зависимость значения $A_{\varphi}$ от времени в разных приёмниках}
	\label{fig:NatA}
\end{figure}

\begin{figure}
	\centering
	\includegraphics[width=0.8\linewidth]{images/Normal_E.png}
	\caption{Зависимость значения $E_{\varphi}$ от времени в разных приёмниках}
	\label{fig:NatE}
\end{figure}


\begin{figure}
	\centering
	\includegraphics[width=0.8\linewidth]{images/Log_A.png}
	\caption{Зависимость значения $A_{\varphi}$ от времени в разных приёмниках (логарифмическая шкала по оси абсцисс)}
	\label{fig:LogA}
\end{figure}

\begin{figure}
	\centering
	\includegraphics[width=0.8\linewidth]{images/Log_E.png}
	\caption{Зависимость значения $E_{\varphi}$ от времени в разных приёмниках (логарифмическая шкала по оси абсцисс)}
	\label{fig:LogE}
\end{figure} 

Как видим, значения вектор-потенциала и электрической напряжённости поля не имеют каких-либо резких колебаний. Из этого можно заключить, что, как и предполагалось, никаких аномальных зон в исследуемой области нет. 

\section{Исследование при разделении нормального и добавочного поля}

Добавим в нашу область аномальный объект со следующими границами: $[-5500; 5500]_x \times [2205; 2355]_y \times [-180; -80]_z$ и значением $\sigma = 4$. Будем искать решение из уравнения на добавочное поле (\ref{eq_1_6}). Получим решения изображенные на рисунках \ref{fig:A_plus_t0} -- \ref{fig:E_plus_t2}.

\begin{figure}
	\centering
	\includegraphics[width=1.0\linewidth]{images/Answer_A_plus_time_layer_1.png}
	\caption{Решение $\overrightarrow{\textbf{A}}^+$ при $t = 1.0с$}
	\label{fig:A_plus_t0}
\end{figure} 


\begin{figure}
	\centering
	\includegraphics[width=1.0\linewidth]{images/Answer_E_plus_time_layer_1.png}
	\caption{Решение $\overrightarrow{\textbf{E}}^+$ при $t = 1.0с$}
	\label{fig:E_plus_t0}
\end{figure} 


\begin{figure}
	\centering
	\includegraphics[width=1.0\linewidth]{images/Answer_A_plus_time_layer_1.0250000000000006.png}
	\caption{Решение $\overrightarrow{\textbf{A}}^+$ при $t = 1.025с$}
	\label{fig:A_plus_t1}
\end{figure} 


\begin{figure}
	\centering
	\includegraphics[width=1.0\linewidth]{images/Answer_E_plus_time_layer_1.0250000000000006.png}
	\caption{Решение $\overrightarrow{\textbf{E}}^+$ при $t = 1.025с$}
	\label{fig:E_plus_t1}
\end{figure} 

\begin{figure}
	\centering
	\includegraphics[width=1.0\linewidth]{images/Answer_A_plus_time_layer_1.05.png}
	\caption{Решение $\overrightarrow{\textbf{A}}^+$ при $t = 1.05с$}
	\label{fig:A_plus_t2}
\end{figure} 


\begin{figure}
	\centering
	\includegraphics[width=1.0\linewidth]{images/Answer_E_plus_time_layer_1.05.png}
	\caption{Решение $\overrightarrow{\textbf{E}}^+$ при $t = 1.05с$}
	\label{fig:E_plus_t2}
\end{figure} 

Рассмотрим для $t = 1.0, t = 1.025, t = 1.05$ значения $\overrightarrow{\textbf{A}}^+$ и $\overrightarrow{\textbf{E}}^+$ на линии, перпендикулярно проходящей к аномальному объекту по оси $y$ при $x = 0.0, z = -130.0$. Получим следующее:

\begin{figure}
	\centering
	\includegraphics[width=0.5\linewidth]{images/Normal_A(y)_1.png}
	\caption{Решение $\overrightarrow{\textbf{A}}^+$ на линии $(0.0, y, -130.0)$ при $t = 1.0с$}
	\label{fig:A_line_t0}
\end{figure} 

\begin{figure}
	\centering
	\includegraphics[width=0.5\linewidth]{images/Normal_E(y)_1.png}
	\caption{Решение $\overrightarrow{\textbf{E}}^+$ на линии $(0.0, y, -130.0)$ при $t = 1.0с$}
	\label{fig:E_line_t0}
\end{figure} 

\begin{figure}
	\centering
	\includegraphics[width=0.5\linewidth]{images/Normal_A(y)_2.png}
	\caption{Решение $\overrightarrow{\textbf{A}}^+$ на линии $(0.0, y, -130.0)$ при $t = 1.025с$}
	\label{fig:A_line_t1}
\end{figure} 

\begin{figure}
	\centering
	\includegraphics[width=0.5\linewidth]{images/Normal_E(y)_2.png}
	\caption{Решение $\overrightarrow{\textbf{E}}^+$ на линии $(0.0, y, -130.0)$  при $t = 1.025с$}
	\label{fig:E_line_t1}
\end{figure} 

\begin{figure}
	\centering
	\includegraphics[width=0.5\linewidth]{images/Normal_A(y)_3.png}
	\caption{Решение $\overrightarrow{\textbf{A}}^+$ на линии $(0.0, y, -130.0)$  при $t = 1.05с$}
	\label{fig:A_line_t2}
\end{figure} 

\begin{figure}
	\centering
	\includegraphics[width=0.5\linewidth]{images/Normal_A(y)_3.png}
	\caption{Решение $\overrightarrow{\textbf{A}}^+$ на линии $(0.0, y, -130.0)$  при $t = 1.05с$}
	\label{fig:E_line_t2}
\end{figure} 


Суммируем полученный результат с нормальным полем и получим состояние поля в разные моменты времени, изображённые на рисунках \ref{fig:A_Istage_t0} -- \ref{fig:E_Istage_t2}.

\begin{figure}
	\centering
	\includegraphics[width=1.0\linewidth]{images/Answer_A_Istage_time_layer_1.png}
	\caption{Решение суммарного поля $\overrightarrow{\textbf{A}}$ при $t = 1.0с$}
	\label{fig:A_Istage_t0}
\end{figure} 


\begin{figure}
	\centering
	\includegraphics[width=1.0\linewidth]{images/Answer_E_Istage_time_layer_1.png}
	\caption{Решение суммарного поля $\overrightarrow{\textbf{E}}$ при $t = 1.0с$}
	\label{fig:E_Istage_t0}
\end{figure} 


\begin{figure}
	\centering
	\includegraphics[width=1.0\linewidth]{images/Answer_A_Istage_time_layer_1.0250000000000006.png}
	\caption{Решение суммарного поля $\overrightarrow{\textbf{A}}$ при $t = 1.025с$}
	\label{fig:A_Istage_t1}
\end{figure} 


\begin{figure}
	\centering
	\includegraphics[width=1.0\linewidth]{images/Answer_E_Istage_time_layer_1.0250000000000006.png}
	\caption{Решение суммарного поля $\overrightarrow{\textbf{E}}$ при $t = 1.025с$}
	\label{fig:E_Istage_t1}
\end{figure} 

\begin{figure}
	\centering
	\includegraphics[width=1.0\linewidth]{images/Answer_A_Istage_time_layer_1.05.png}
	\caption{Решение суммарного поля $\overrightarrow{\textbf{A}}$ при $t = 1.05с$}
	\label{fig:A_Istage_t2}
\end{figure} 


\begin{figure}
	\centering
	\includegraphics[width=1.0\linewidth]{images/Answer_E_Istage_time_layer_1.05.png}
	\caption{Решение суммарного поля $\overrightarrow{\textbf{E}}$ при $t = 1.05с$}
	\label{fig:E_Istage_t2}
\end{figure} 

Полученные значения \ref{fig:A_Log_added} -- \ref{fig:E_Log_added} $\overrightarrow{\textbf{A}}$ и $\overrightarrow{\textbf{E}}$ рассмотрим на приёмниках.

\begin{figure}
	\centering
	\includegraphics[width=0.8\linewidth]{images/Log_A_obj1.png}
	\caption{Решение суммарного поля $\overrightarrow{\textbf{E}}$ при $t = 1.05с$}
	\label{fig:A_Log_added}
\end{figure} 


\begin{figure}
	\centering
	\includegraphics[width=0.8\linewidth]{images/Log_E_obj1.png}
	\caption{Решение суммарного поля $\overrightarrow{\textbf{E}}$ при $t = 1.05с$}
	\label{fig:E_Log_added}
\end{figure} 

Сравнивая показатели на приёмниках до добавления аномалии \ref{fig:LogA} -- \ref{fig:LogE} и после \ref{fig:A_Log_added} -- \ref{fig:A_Log_added}, можно заметить, что значения напряжённости электрического поля на красном приёмнике претерпели наиболее сильные изменения, т.к. он стал более похожим на гиперболу, нежели прямую линию. Значения на синем приёмнике начали изменяться уже в последние сотые секунды исследования. 

\section{Исследование многоэтапного разделения нормального и добавочных полей}

Добавим ещё один аномальный объект со следующими границами: $[-1305; -1050]_x \times [-2255; -2178]_y \times [-1250; -800]_z$ и значением $\sigma = 17$. Будем искать решение из уравнения на добавочное поле (\ref{eq_1_6}). Получим решения в разные моменты времени изображенные на рисунках \ref{fig:A_2plus_t0} -- \ref{fig:E_2plus_t2}.

\begin{figure}
	\centering
	\includegraphics[width=1.0\linewidth]{images/Answer_A_2plus_time_layer_1.png}
	\caption{Решение $\overrightarrow{\textbf{A}}^+$ при $t = 1.0с$}
	\label{fig:A_2plus_t0}
\end{figure} 


\begin{figure}
	\centering
	\includegraphics[width=1.0\linewidth]{images/Answer_E_2plus_time_layer_1.png}
	\caption{Решение $\overrightarrow{\textbf{E}}^+$ при $t = 1.0с$}
	\label{fig:E_2plus_t0}
\end{figure} 


\begin{figure}
	\centering
	\includegraphics[width=1.0\linewidth]{images/Answer_A_2plus_time_layer_1.0250000000000006.png}
	\caption{Решение $\overrightarrow{\textbf{A}}^+$ при $t = 1.025с$}
	\label{fig:A_2plus_t1}
\end{figure} 


\begin{figure}
	\centering
	\includegraphics[width=1.0\linewidth]{images/Answer_E_2plus_time_layer_1.0250000000000006.png}
	\caption{Решение $\overrightarrow{\textbf{E}}^+$ при $t = 1.025с$}
	\label{fig:E_2plus_t1}
\end{figure} 

\begin{figure}
	\centering
	\includegraphics[width=1.0\linewidth]{images/Answer_A_2plus_time_layer_1.05.png}
	\caption{Решение $\overrightarrow{\textbf{A}}^+$ при $t = 1.05с$}
	\label{fig:A_2plus_t2}
\end{figure} 


\begin{figure}
	\centering
	\includegraphics[width=1.0\linewidth]{images/Answer_E_2plus_time_layer_1.05.png}
	\caption{Решение $\overrightarrow{\textbf{E}}^+$ при $t = 1.05с$}
	\label{fig:E_2plus_t2}
\end{figure} 

Рассмотрим для $t = 1.0, t = 1.025, t = 1.05$ значения $\overrightarrow{\textbf{A}}^+$ и $\overrightarrow{\textbf{E}}^+$ на линии, перпендикулярно проходящей к аномальному объекту по оси $y$ при $x = -1177.5, z = -1050.0$. Получим следующее:

\begin{figure}
	\centering
	\includegraphics[width=0.5\linewidth]{images/Normal_A_obj2_1.png}
	\caption{Решение $\overrightarrow{\textbf{A}}$ на линии $(0.0, y, -130.0)$ при $t = 1.025с$}
	\label{fig:A_2line_t0}
\end{figure} 

\begin{figure}
	\centering
	\includegraphics[width=0.5\linewidth]{images/Normal_E_obj2_1.png}
	\caption{Решение $\overrightarrow{\textbf{E}}$ на линии $(0.0, y, -130.0)$ при $t = 1.025с$}
	\label{fig:E_2line_t0}
\end{figure} 

\begin{figure}
	\centering
	\includegraphics[width=0.5\linewidth]{images/Normal_A_obj2_2.png}
	\caption{Решение $\overrightarrow{\textbf{A}}$ на линии $(0.0, y, -130.0)$ при $t = 1.025с$}
	\label{fig:A_2line_t1}
\end{figure} 

\begin{figure}
	\centering
	\includegraphics[width=0.5\linewidth]{images/Normal_E_obj2_2.png}
	\caption{Решение $\overrightarrow{\textbf{E}}$ на линии $(0.0, y, -130.0)$  при $t = 1.025с$}
	\label{fig:E_2line_t1}
\end{figure} 

\begin{figure}
	\centering
	\includegraphics[width=0.5\linewidth]{images/Normal_A_obj2_3.png}
	\caption{Решение $\overrightarrow{\textbf{A}}$ на линии $(0.0, y, -130.0)$  при $t = 1.05с$}
	\label{fig:A_2line_t2}
\end{figure} 

\begin{figure}
	\centering
	\includegraphics[width=0.5\linewidth]{images/Normal_E_obj2_2.png}
	\caption{Решение $\overrightarrow{\textbf{A}}$ на линии $(0.0, y, -130.0)$  при $t = 1.05с$}
	\label{fig:E_2line_t2}
\end{figure} 

Суммируем полученный результат с полем без аномалий и получим состояние поля в разные моменты времени, изображённые на рисунках \ref{fig:A_IIstage_t0} -- \ref{fig:E_IIstage_t2}.

\begin{figure}
	\centering
	\includegraphics[width=1.0\linewidth]{images/Answer_A_IIstage_time_layer_1.png}
	\caption{Решение суммарного поля $\overrightarrow{\textbf{A}}$ при $t = 1.0с$}
	\label{fig:A_IIstage_t0}
\end{figure} 


\begin{figure}
	\centering
	\includegraphics[width=1.0\linewidth]{images/Answer_E_IIstage_time_layer_1.png}
	\caption{Решение суммарного поля $\overrightarrow{\textbf{E}}$ при $t = 1.0с$}
	\label{fig:E_IIstage_t0}
\end{figure} 


\begin{figure}
	\centering
	\includegraphics[width=1.0\linewidth]{images/Answer_A_IIstage_time_layer_1.0250000000000006.png}
	\caption{Решение суммарного поля $\overrightarrow{\textbf{A}}$ при $t = 1.025с$}
	\label{fig:A_IIstage_t1}
\end{figure} 


\begin{figure}
	\centering
	\includegraphics[width=1.0\linewidth]{images/Answer_E_IIstage_time_layer_1.0250000000000006.png}
	\caption{Решение суммарного поля $\overrightarrow{\textbf{E}}$ при $t = 1.025с$}
	\label{fig:E_IIstage_t1}
\end{figure} 

\begin{figure}
	\centering
	\includegraphics[width=1.0\linewidth]{images/Answer_A_IIstage_time_layer_1.05.png}
	\caption{Решение суммарного поля $\overrightarrow{\textbf{A}}$ при $t = 1.05с$}
	\label{fig:A_IIstage_t2}
\end{figure} 


\begin{figure}
	\centering
	\includegraphics[width=1.0\linewidth]{images/Answer_E_IIstage_time_layer_1.05.png}
	\caption{Решение суммарного поля $\overrightarrow{\textbf{E}}$ при $t = 1.05с$}
	\label{fig:E_IIstage_t2}
\end{figure} 

Полученные значения \ref{fig:A_Log_added2} -- \ref{fig:E_Log_added2} $\overrightarrow{\textbf{A}}$ и $\overrightarrow{\textbf{E}}$ рассмотрим на приёмниках.

\begin{figure}
	\centering
	\includegraphics[width=0.8\linewidth]{images/Log_A_obj2.png}
	\caption{Решение суммарного поля $\overrightarrow{\textbf{E}}$ при $t = 1.05с$}
	\label{fig:A_Log_added2}
\end{figure} 


\begin{figure}
	\centering
	\includegraphics[width=0.8\linewidth]{images/Log_E_obj2.png}
	\caption{Решение суммарного поля $\overrightarrow{\textbf{E}}$ при $t = 1.05с$}
	\label{fig:E_Log_added2}
\end{figure} 

Сравнивая показатели на приёмниках до добавления аномалии \ref{fig:LogA} -- \ref{fig:LogE} и после \ref{fig:A_Log_added2} -- \ref{fig:A_Log_added2}, можно заметить, что значения напряжённости электрического поля ни на одном из приёмников не претерпели изменения, по всей видимости из-за неудачного их расположения. Для изменения ситуации необходимо было бы увеличить размеры расчётной области по пространству и временного диапазона. 

Рассмотрим теперь поле с двумя объектами сразу. Для этого сначала добавим первый объект к чистой расчётной области, после используя его в качестве нормального, добавим вторую аномалию.

\begin{figure}
	\centering
	\includegraphics[width=1.0\linewidth]{images/Answer_A_both_time_layer_1.png}
	\caption{Решение суммарного поля $\overrightarrow{\textbf{A}}$ при $t = 1.0с$}
	\label{fig:A_both_t0}
\end{figure} 


\begin{figure}
	\centering
	\includegraphics[width=1.0\linewidth]{images/Answer_E_both_time_layer_1.png}
	\caption{Решение суммарного поля $\overrightarrow{\textbf{E}}$ при $t = 1.0с$}
	\label{fig:E_both_t0}
\end{figure} 


\begin{figure}
	\centering
	\includegraphics[width=1.0\linewidth]{images/Answer_A_both_time_layer_1.0250000000000006.png}
	\caption{Решение суммарного поля $\overrightarrow{\textbf{A}}$ при $t = 1.025с$}
	\label{fig:A_both_t1}
\end{figure} 


\begin{figure}
	\centering
	\includegraphics[width=1.0\linewidth]{images/Answer_E_both_time_layer_1.0250000000000006.png}
	\caption{Решение суммарного поля $\overrightarrow{\textbf{E}}$ при $t = 1.025с$}
	\label{fig:E_both_t1}
\end{figure} 

\begin{figure}
	\centering
	\includegraphics[width=1.0\linewidth]{images/Answer_A_both_time_layer_1.05.png}
	\caption{Решение суммарного поля $\overrightarrow{\textbf{A}}$ при $t = 1.05с$}
	\label{fig:A_both_t2}
\end{figure} 


\begin{figure}
	\centering
	\includegraphics[width=1.0\linewidth]{images/Answer_E_both_time_layer_1.05.png}
	\caption{Решение суммарного поля $\overrightarrow{\textbf{E}}$ при $t = 1.05с$}
	\label{fig:E_both_t2}
\end{figure} 

Полученные значения \ref{fig:A_Log_both} -- \ref{fig:E_Log_both} $\overrightarrow{\textbf{A}}$ и $\overrightarrow{\textbf{E}}$ рассмотрим на приёмниках.

\begin{figure}
	\centering
	\includegraphics[width=0.8\linewidth]{images/Log_A_both.png}
	\caption{Решение суммарного поля $\overrightarrow{\textbf{E}}$ при $t = 1.05с$}
	\label{fig:A_Log_both}
\end{figure} 


\begin{figure}
	\centering
	\includegraphics[width=0.8\linewidth]{images/Log_E_both.png}
	\caption{Решение суммарного поля $\overrightarrow{\textbf{E}}$ при $t = 1.05с$}
	\label{fig:E_Log_both}
\end{figure} 

Сравнивая показатели на приёмниках до добавления обеих аномалий \ref{fig:LogA} -- \ref{fig:LogE} и после \ref{fig:A_Log_both} -- \ref{fig:E_Log_both}, можно заметить, что значения напряжённости электрического поля на красном приёмнике все те же изменения, что и без учёта второго объекта. Это связано со слабым влиянием второго объекта на приёмники. 
\chapter{Теоретическая часть}

\section{Многоэтапная схема разделения поля}

При решении многих векторных электромагнитных задач существенного повышения точности получаемого решения можно добиться в результате использования методов, основанных на разделении из искомого поля достаточно близкого к нему поля меньшей размерности. Сначала выделяется поле, создаваемое в среде, максимально упрощённой относительно исходной. Её решение берётся в качестве основного поля первого уровня. На базе этого поля формируется задача на добавочное поле, в которую включается часть неоднородностей исходной задачи, дающих максимальный вклад в искомое решение. Используемая для нахождения этого добавочного поля сетка строится так, чтобы максимально учесть влияние источников, порождённых включёнными в на этом этапе неоднородностями.

Далее в качестве основного поля будет учитываться сумма основного и добавочного на предыдущем этапе выделения. Новое добавочное поле будет формироваться из учёта следующих по влиянию на решение исходной задачи. Процесс можно продолжать до тех пор, пока не будут учтены все неоднородности среды.

Если рассматривать уравнение (\ref{eq_1_6}), то выходит, что излучаемое электромагнитное поле полностью описывается вектором-потенциалом $\overrightarrow{\textbf{A}}$, где значения магнитной индукции и электрической напряжённости определяются, как $\overrightarrow{\textbf{B}} = \text{rot} \overrightarrow{\textbf{A}}$ и $\overrightarrow{\textbf{E}} = -\frac{\partial \overrightarrow{\textbf{A}}}{\partial t}$ соответственно. Тогда каждое из полей $\overrightarrow{\textbf{A}}^{0}$ (нормальное) и $\overrightarrow{\textbf{A}}^{+}$ (добавочное) будет определять значения магнитной индукции и напряжённости электрического поля: $\overrightarrow{\textbf{B}}^0 = \text{rot} \overrightarrow{\textbf{A}}^0$ и $\overrightarrow{\textbf{E}}^0 = -\frac{\partial \overrightarrow{\textbf{A}}^0}{\partial t}$ для нормального, $\overrightarrow{\textbf{B}}^+ = \text{rot} \overrightarrow{\textbf{A}}^+$ и $\overrightarrow{\textbf{E}}^+ = -\frac{\partial \overrightarrow{\textbf{A}}^+}{\partial t}$ для добавочного. В нашем случае решение $\overrightarrow{\textbf{A}}^0$ получено из решения скалярной осесимметричной задачи (\ref{eq_1_5}).

Найдём вариационную постановку для уравнения ($\ref{eq_1_6}$):

\begin{equation} \label{eq_2_1}
	\frac{1}{\mu_0} \int \limits_{\Omega} \text{rot} \overrightarrow{\textbf{A}}^+ \text{rot} \overrightarrow{\Psi} d \Omega + \int \limits_{\Omega} \sigma \frac{\partial \overrightarrow{\textbf{A}}^+}{\partial t} \overrightarrow{\Psi} d \Omega = \int \limits_{\Omega}(\sigma - \sigma_n) \overrightarrow{\textbf{E}^0} \overrightarrow{\Psi} d \Omega.
\end{equation}

Из (\ref{eq_2_1}) получим матричное уравнение для добавочного поля:

\begin{equation} \label{eq_2_2}
	\left(\frac{1}{\mu_0} \hat{\textbf{G}} + \sigma \frac{1}{\Delta t} \hat{\textbf{M}} \right) \text{q}^i = (\sigma - \sigma_n) \overrightarrow{\textbf{E}}^0 \hat{\textbf{M}} + \sigma \frac{1}{\Delta t} \hat{\textbf{M}}\text{q}^{i-1}.
\end{equation}
\chapter{Исследования}

\section{Исследование первичного поля}

Пусть источник индукционного поля лежит на расстоянии $R = 500$ м от оси симметрии и имееет силу тока, равную $J_{\varphi} = 1.0$ А. Также условимся, что источник работал достаточно долго, чтобы создать стабильное электромагнитное поле. Сетка по времени равномерная: $t=[1.0; 1.05]$ на 200 временных слоёв. После истечения первой секунды мы отключим наш источник, т.е. $J_{\varphi} = 0.0$ A при $t > 1.0$. На рисунках \ref{fig:A_phi_0} -- \ref{fig:E_phi_2} представлено распространение этого поля в среде в начальный, промежуточных и последний момент времени.

\begin{figure}
	\centering
	\includegraphics[width=1.0\linewidth]{images/Answer_A_time_layer_1.png}
	\caption{Решение $A_{\varphi}$ при $t = 1.0с$}
	\label{fig:A_phi_0}
\end{figure}

\begin{figure}
	\centering
	\includegraphics[width=1.0\linewidth]{images/Answer_E_time_layer_1.png}
	\caption{Решение $E_{\varphi}$ при $t = 1.0с$}
	\label{fig:E_phi_0}
\end{figure}

\begin{figure}
	\centering
	\includegraphics[width=1.0\linewidth]{images/Answer_A_time_layer_1.0250000000000083.png}
	\caption{Решение $A_{\varphi}$ при $t = 1.025с$}
	\label{fig:A_phi_1}
\end{figure}

\begin{figure}
	\centering
	\includegraphics[width=1.0\linewidth]{images/Answer_E_time_layer_1.0250000000000083.png}
	\caption{Решение $E_{\varphi}$ при $t = 1.025с$}
	\label{fig:E_phi_1}
\end{figure} 

\begin{figure}
	\centering
	\includegraphics[width=1.0\linewidth]{images/Answer_A_time_layer_1.05.png}
	\caption{Решение $A_{\varphi}$ при $t = 1.05с$}
	\label{fig:A_phi_2}
\end{figure}

\begin{figure}
	\centering
	\includegraphics[width=1.0\linewidth]{images/Answer_E_time_layer_1.05.png}
	\caption{Решение $A_{\varphi}$ при $t = 1.05с$}
	\label{fig:E_phi_2}
\end{figure} 

Расположим на расчётной области приёмники в каждой горизонтально-слоистой среде и проведем замеры значений вектор-потенциала и электрического поля в точках $(2500; 0; -100)$, $(2500; 0; -200)$, $(10; 0; -700)$, $(1000; 0; -1250)$. Отобразим на графиках \ref{fig:NatA} -- \ref{fig:LogE} полученные значения.

\begin{figure}
	\centering
	\includegraphics[width=0.8\linewidth]{images/Normal_A.png}
	\caption{Зависимость значения $A_{\varphi}$ от времени в разных приёмниках}
	\label{fig:NatA}
\end{figure}

\begin{figure}
	\centering
	\includegraphics[width=0.8\linewidth]{images/Normal_E.png}
	\caption{Зависимость значения $E_{\varphi}$ от времени в разных приёмниках}
	\label{fig:NatE}
\end{figure}


\begin{figure}
	\centering
	\includegraphics[width=0.8\linewidth]{images/Log_A.png}
	\caption{Зависимость значения $A_{\varphi}$ от времени в разных приёмниках (логарифмическая шкала по оси абсцисс)}
	\label{fig:LogA}
\end{figure}

\begin{figure}
	\centering
	\includegraphics[width=0.8\linewidth]{images/Log_E.png}
	\caption{Зависимость значения $E_{\varphi}$ от времени в разных приёмниках (логарифмическая шкала по оси абсцисс)}
	\label{fig:LogE}
\end{figure} 

Как видим, значения вектор-потенциала и электрической напряжённости поля не имеют каких-либо резких колебаний. Из этого можно заключить, что, как и предполагалось, никаких аномальных зон в исследуемой области нет. 

\section{Исследование при разделении нормального и добавочного поля}

Добавим в нашу область аномальный объект со следующими границами: $[-5500; 5500]_x \times [2205; 2355]_y \times [-180; -80]_z$ и значением $\sigma = 4$. Будем искать решение из уравнения на добавочное поле (\ref{eq_1_6}). Получим решения изображенные на рисунках \ref{fig:A_plus_t0} -- \ref{fig:E_plus_t2}.

\begin{figure}
	\centering
	\includegraphics[width=1.0\linewidth]{images/Answer_A_plus_time_layer_1.png}
	\caption{Решение $\overrightarrow{\textbf{A}}^+$ при $t = 1.0с$}
	\label{fig:A_plus_t0}
\end{figure} 


\begin{figure}
	\centering
	\includegraphics[width=1.0\linewidth]{images/Answer_E_plus_time_layer_1.png}
	\caption{Решение $\overrightarrow{\textbf{E}}^+$ при $t = 1.0с$}
	\label{fig:E_plus_t0}
\end{figure} 


\begin{figure}
	\centering
	\includegraphics[width=1.0\linewidth]{images/Answer_A_plus_time_layer_1.0250000000000006.png}
	\caption{Решение $\overrightarrow{\textbf{A}}^+$ при $t = 1.025с$}
	\label{fig:A_plus_t1}
\end{figure} 


\begin{figure}
	\centering
	\includegraphics[width=1.0\linewidth]{images/Answer_E_plus_time_layer_1.0250000000000006.png}
	\caption{Решение $\overrightarrow{\textbf{E}}^+$ при $t = 1.025с$}
	\label{fig:E_plus_t1}
\end{figure} 

\begin{figure}
	\centering
	\includegraphics[width=1.0\linewidth]{images/Answer_A_plus_time_layer_1.05.png}
	\caption{Решение $\overrightarrow{\textbf{A}}^+$ при $t = 1.05с$}
	\label{fig:A_plus_t2}
\end{figure} 


\begin{figure}
	\centering
	\includegraphics[width=1.0\linewidth]{images/Answer_E_plus_time_layer_1.05.png}
	\caption{Решение $\overrightarrow{\textbf{E}}^+$ при $t = 1.05с$}
	\label{fig:E_plus_t2}
\end{figure} 

Рассмотрим для $t = 1.0, t = 1.025, t = 1.05$ значения $\overrightarrow{\textbf{A}}^+$ и $\overrightarrow{\textbf{E}}^+$ на линии, перпендикулярно проходящей к аномальному объекту по оси $y$ при $x = 0.0, z = -130.0$. Получим следующее:

\begin{figure}
	\centering
	\includegraphics[width=0.5\linewidth]{images/Normal_A(y)_1.png}
	\caption{Решение $\overrightarrow{\textbf{A}}^+$ на линии $(0.0, y, -130.0)$ при $t = 1.0с$}
	\label{fig:A_line_t0}
\end{figure} 

\begin{figure}
	\centering
	\includegraphics[width=0.5\linewidth]{images/Normal_E(y)_1.png}
	\caption{Решение $\overrightarrow{\textbf{E}}^+$ на линии $(0.0, y, -130.0)$ при $t = 1.0с$}
	\label{fig:E_line_t0}
\end{figure} 

\begin{figure}
	\centering
	\includegraphics[width=0.5\linewidth]{images/Normal_A(y)_2.png}
	\caption{Решение $\overrightarrow{\textbf{A}}^+$ на линии $(0.0, y, -130.0)$ при $t = 1.025с$}
	\label{fig:A_line_t1}
\end{figure} 

\begin{figure}
	\centering
	\includegraphics[width=0.5\linewidth]{images/Normal_E(y)_2.png}
	\caption{Решение $\overrightarrow{\textbf{E}}^+$ на линии $(0.0, y, -130.0)$  при $t = 1.025с$}
	\label{fig:E_line_t1}
\end{figure} 

\begin{figure}
	\centering
	\includegraphics[width=0.5\linewidth]{images/Normal_A(y)_3.png}
	\caption{Решение $\overrightarrow{\textbf{A}}^+$ на линии $(0.0, y, -130.0)$  при $t = 1.05с$}
	\label{fig:A_line_t2}
\end{figure} 

\begin{figure}
	\centering
	\includegraphics[width=0.5\linewidth]{images/Normal_A(y)_3.png}
	\caption{Решение $\overrightarrow{\textbf{A}}^+$ на линии $(0.0, y, -130.0)$  при $t = 1.05с$}
	\label{fig:E_line_t2}
\end{figure} 


Суммируем полученный результат с нормальным полем и получим состояние поля в разные моменты времени, изображённые на рисунках \ref{fig:A_Istage_t0} -- \ref{fig:E_Istage_t2}.

\begin{figure}
	\centering
	\includegraphics[width=1.0\linewidth]{images/Answer_A_Istage_time_layer_1.png}
	\caption{Решение суммарного поля $\overrightarrow{\textbf{A}}$ при $t = 1.0с$}
	\label{fig:A_Istage_t0}
\end{figure} 


\begin{figure}
	\centering
	\includegraphics[width=1.0\linewidth]{images/Answer_E_Istage_time_layer_1.png}
	\caption{Решение суммарного поля $\overrightarrow{\textbf{E}}$ при $t = 1.0с$}
	\label{fig:E_Istage_t0}
\end{figure} 


\begin{figure}
	\centering
	\includegraphics[width=1.0\linewidth]{images/Answer_A_Istage_time_layer_1.0250000000000006.png}
	\caption{Решение суммарного поля $\overrightarrow{\textbf{A}}$ при $t = 1.025с$}
	\label{fig:A_Istage_t1}
\end{figure} 


\begin{figure}
	\centering
	\includegraphics[width=1.0\linewidth]{images/Answer_E_Istage_time_layer_1.0250000000000006.png}
	\caption{Решение суммарного поля $\overrightarrow{\textbf{E}}$ при $t = 1.025с$}
	\label{fig:E_Istage_t1}
\end{figure} 

\begin{figure}
	\centering
	\includegraphics[width=1.0\linewidth]{images/Answer_A_Istage_time_layer_1.05.png}
	\caption{Решение суммарного поля $\overrightarrow{\textbf{A}}$ при $t = 1.05с$}
	\label{fig:A_Istage_t2}
\end{figure} 


\begin{figure}
	\centering
	\includegraphics[width=1.0\linewidth]{images/Answer_E_Istage_time_layer_1.05.png}
	\caption{Решение суммарного поля $\overrightarrow{\textbf{E}}$ при $t = 1.05с$}
	\label{fig:E_Istage_t2}
\end{figure} 

Полученные значения \ref{fig:A_Log_added} -- \ref{fig:E_Log_added} $\overrightarrow{\textbf{A}}$ и $\overrightarrow{\textbf{E}}$ рассмотрим на приёмниках.

\begin{figure}
	\centering
	\includegraphics[width=0.8\linewidth]{images/Log_A_obj1.png}
	\caption{Решение суммарного поля $\overrightarrow{\textbf{E}}$ при $t = 1.05с$}
	\label{fig:A_Log_added}
\end{figure} 


\begin{figure}
	\centering
	\includegraphics[width=0.8\linewidth]{images/Log_E_obj1.png}
	\caption{Решение суммарного поля $\overrightarrow{\textbf{E}}$ при $t = 1.05с$}
	\label{fig:E_Log_added}
\end{figure} 

Сравнивая показатели на приёмниках до добавления аномалии \ref{fig:LogA} -- \ref{fig:LogE} и после \ref{fig:A_Log_added} -- \ref{fig:A_Log_added}, можно заметить, что значения напряжённости электрического поля на красном приёмнике претерпели наиболее сильные изменения, т.к. он стал более похожим на гиперболу, нежели прямую линию. Значения на синем приёмнике начали изменяться уже в последние сотые секунды исследования. 

\section{Исследование многоэтапного разделения нормального и добавочных полей}

Добавим ещё один аномальный объект со следующими границами: $[-1305; -1050]_x \times [-2255; -2178]_y \times [-1250; -800]_z$ и значением $\sigma = 17$. Будем искать решение из уравнения на добавочное поле (\ref{eq_1_6}). Получим решения в разные моменты времени изображенные на рисунках \ref{fig:A_2plus_t0} -- \ref{fig:E_2plus_t2}.

\begin{figure}
	\centering
	\includegraphics[width=1.0\linewidth]{images/Answer_A_2plus_time_layer_1.png}
	\caption{Решение $\overrightarrow{\textbf{A}}^+$ при $t = 1.0с$}
	\label{fig:A_2plus_t0}
\end{figure} 


\begin{figure}
	\centering
	\includegraphics[width=1.0\linewidth]{images/Answer_E_2plus_time_layer_1.png}
	\caption{Решение $\overrightarrow{\textbf{E}}^+$ при $t = 1.0с$}
	\label{fig:E_2plus_t0}
\end{figure} 


\begin{figure}
	\centering
	\includegraphics[width=1.0\linewidth]{images/Answer_A_2plus_time_layer_1.0250000000000006.png}
	\caption{Решение $\overrightarrow{\textbf{A}}^+$ при $t = 1.025с$}
	\label{fig:A_2plus_t1}
\end{figure} 


\begin{figure}
	\centering
	\includegraphics[width=1.0\linewidth]{images/Answer_E_2plus_time_layer_1.0250000000000006.png}
	\caption{Решение $\overrightarrow{\textbf{E}}^+$ при $t = 1.025с$}
	\label{fig:E_2plus_t1}
\end{figure} 

\begin{figure}
	\centering
	\includegraphics[width=1.0\linewidth]{images/Answer_A_2plus_time_layer_1.05.png}
	\caption{Решение $\overrightarrow{\textbf{A}}^+$ при $t = 1.05с$}
	\label{fig:A_2plus_t2}
\end{figure} 


\begin{figure}
	\centering
	\includegraphics[width=1.0\linewidth]{images/Answer_E_2plus_time_layer_1.05.png}
	\caption{Решение $\overrightarrow{\textbf{E}}^+$ при $t = 1.05с$}
	\label{fig:E_2plus_t2}
\end{figure} 

Рассмотрим для $t = 1.0, t = 1.025, t = 1.05$ значения $\overrightarrow{\textbf{A}}^+$ и $\overrightarrow{\textbf{E}}^+$ на линии, перпендикулярно проходящей к аномальному объекту по оси $y$ при $x = -1177.5, z = -1050.0$. Получим следующее:

\begin{figure}
	\centering
	\includegraphics[width=0.5\linewidth]{images/Normal_A_obj2_1.png}
	\caption{Решение $\overrightarrow{\textbf{A}}$ на линии $(0.0, y, -130.0)$ при $t = 1.025с$}
	\label{fig:A_2line_t0}
\end{figure} 

\begin{figure}
	\centering
	\includegraphics[width=0.5\linewidth]{images/Normal_E_obj2_1.png}
	\caption{Решение $\overrightarrow{\textbf{E}}$ на линии $(0.0, y, -130.0)$ при $t = 1.025с$}
	\label{fig:E_2line_t0}
\end{figure} 

\begin{figure}
	\centering
	\includegraphics[width=0.5\linewidth]{images/Normal_A_obj2_2.png}
	\caption{Решение $\overrightarrow{\textbf{A}}$ на линии $(0.0, y, -130.0)$ при $t = 1.025с$}
	\label{fig:A_2line_t1}
\end{figure} 

\begin{figure}
	\centering
	\includegraphics[width=0.5\linewidth]{images/Normal_E_obj2_2.png}
	\caption{Решение $\overrightarrow{\textbf{E}}$ на линии $(0.0, y, -130.0)$  при $t = 1.025с$}
	\label{fig:E_2line_t1}
\end{figure} 

\begin{figure}
	\centering
	\includegraphics[width=0.5\linewidth]{images/Normal_A_obj2_3.png}
	\caption{Решение $\overrightarrow{\textbf{A}}$ на линии $(0.0, y, -130.0)$  при $t = 1.05с$}
	\label{fig:A_2line_t2}
\end{figure} 

\begin{figure}
	\centering
	\includegraphics[width=0.5\linewidth]{images/Normal_E_obj2_2.png}
	\caption{Решение $\overrightarrow{\textbf{A}}$ на линии $(0.0, y, -130.0)$  при $t = 1.05с$}
	\label{fig:E_2line_t2}
\end{figure} 

Суммируем полученный результат с полем без аномалий и получим состояние поля в разные моменты времени, изображённые на рисунках \ref{fig:A_IIstage_t0} -- \ref{fig:E_IIstage_t2}.

\begin{figure}
	\centering
	\includegraphics[width=1.0\linewidth]{images/Answer_A_IIstage_time_layer_1.png}
	\caption{Решение суммарного поля $\overrightarrow{\textbf{A}}$ при $t = 1.0с$}
	\label{fig:A_IIstage_t0}
\end{figure} 


\begin{figure}
	\centering
	\includegraphics[width=1.0\linewidth]{images/Answer_E_IIstage_time_layer_1.png}
	\caption{Решение суммарного поля $\overrightarrow{\textbf{E}}$ при $t = 1.0с$}
	\label{fig:E_IIstage_t0}
\end{figure} 


\begin{figure}
	\centering
	\includegraphics[width=1.0\linewidth]{images/Answer_A_IIstage_time_layer_1.0250000000000006.png}
	\caption{Решение суммарного поля $\overrightarrow{\textbf{A}}$ при $t = 1.025с$}
	\label{fig:A_IIstage_t1}
\end{figure} 


\begin{figure}
	\centering
	\includegraphics[width=1.0\linewidth]{images/Answer_E_IIstage_time_layer_1.0250000000000006.png}
	\caption{Решение суммарного поля $\overrightarrow{\textbf{E}}$ при $t = 1.025с$}
	\label{fig:E_IIstage_t1}
\end{figure} 

\begin{figure}
	\centering
	\includegraphics[width=1.0\linewidth]{images/Answer_A_IIstage_time_layer_1.05.png}
	\caption{Решение суммарного поля $\overrightarrow{\textbf{A}}$ при $t = 1.05с$}
	\label{fig:A_IIstage_t2}
\end{figure} 


\begin{figure}
	\centering
	\includegraphics[width=1.0\linewidth]{images/Answer_E_IIstage_time_layer_1.05.png}
	\caption{Решение суммарного поля $\overrightarrow{\textbf{E}}$ при $t = 1.05с$}
	\label{fig:E_IIstage_t2}
\end{figure} 

Полученные значения \ref{fig:A_Log_added2} -- \ref{fig:E_Log_added2} $\overrightarrow{\textbf{A}}$ и $\overrightarrow{\textbf{E}}$ рассмотрим на приёмниках.

\begin{figure}
	\centering
	\includegraphics[width=0.8\linewidth]{images/Log_A_obj2.png}
	\caption{Решение суммарного поля $\overrightarrow{\textbf{E}}$ при $t = 1.05с$}
	\label{fig:A_Log_added2}
\end{figure} 


\begin{figure}
	\centering
	\includegraphics[width=0.8\linewidth]{images/Log_E_obj2.png}
	\caption{Решение суммарного поля $\overrightarrow{\textbf{E}}$ при $t = 1.05с$}
	\label{fig:E_Log_added2}
\end{figure} 

Сравнивая показатели на приёмниках до добавления аномалии \ref{fig:LogA} -- \ref{fig:LogE} и после \ref{fig:A_Log_added2} -- \ref{fig:A_Log_added2}, можно заметить, что значения напряжённости электрического поля ни на одном из приёмников не претерпели изменения, по всей видимости из-за неудачного их расположения. Для изменения ситуации необходимо было бы увеличить размеры расчётной области по пространству и временного диапазона. 

Рассмотрим теперь поле с двумя объектами сразу. Для этого сначала добавим первый объект к чистой расчётной области, после используя его в качестве нормального, добавим вторую аномалию.

\begin{figure}
	\centering
	\includegraphics[width=1.0\linewidth]{images/Answer_A_both_time_layer_1.png}
	\caption{Решение суммарного поля $\overrightarrow{\textbf{A}}$ при $t = 1.0с$}
	\label{fig:A_both_t0}
\end{figure} 


\begin{figure}
	\centering
	\includegraphics[width=1.0\linewidth]{images/Answer_E_both_time_layer_1.png}
	\caption{Решение суммарного поля $\overrightarrow{\textbf{E}}$ при $t = 1.0с$}
	\label{fig:E_both_t0}
\end{figure} 


\begin{figure}
	\centering
	\includegraphics[width=1.0\linewidth]{images/Answer_A_both_time_layer_1.0250000000000006.png}
	\caption{Решение суммарного поля $\overrightarrow{\textbf{A}}$ при $t = 1.025с$}
	\label{fig:A_both_t1}
\end{figure} 


\begin{figure}
	\centering
	\includegraphics[width=1.0\linewidth]{images/Answer_E_both_time_layer_1.0250000000000006.png}
	\caption{Решение суммарного поля $\overrightarrow{\textbf{E}}$ при $t = 1.025с$}
	\label{fig:E_both_t1}
\end{figure} 

\begin{figure}
	\centering
	\includegraphics[width=1.0\linewidth]{images/Answer_A_both_time_layer_1.05.png}
	\caption{Решение суммарного поля $\overrightarrow{\textbf{A}}$ при $t = 1.05с$}
	\label{fig:A_both_t2}
\end{figure} 


\begin{figure}
	\centering
	\includegraphics[width=1.0\linewidth]{images/Answer_E_both_time_layer_1.05.png}
	\caption{Решение суммарного поля $\overrightarrow{\textbf{E}}$ при $t = 1.05с$}
	\label{fig:E_both_t2}
\end{figure} 

Полученные значения \ref{fig:A_Log_both} -- \ref{fig:E_Log_both} $\overrightarrow{\textbf{A}}$ и $\overrightarrow{\textbf{E}}$ рассмотрим на приёмниках.

\begin{figure}
	\centering
	\includegraphics[width=0.8\linewidth]{images/Log_A_both.png}
	\caption{Решение суммарного поля $\overrightarrow{\textbf{E}}$ при $t = 1.05с$}
	\label{fig:A_Log_both}
\end{figure} 


\begin{figure}
	\centering
	\includegraphics[width=0.8\linewidth]{images/Log_E_both.png}
	\caption{Решение суммарного поля $\overrightarrow{\textbf{E}}$ при $t = 1.05с$}
	\label{fig:E_Log_both}
\end{figure} 

Сравнивая показатели на приёмниках до добавления обеих аномалий \ref{fig:LogA} -- \ref{fig:LogE} и после \ref{fig:A_Log_both} -- \ref{fig:E_Log_both}, можно заметить, что значения напряжённости электрического поля на красном приёмнике все те же изменения, что и без учёта второго объекта. Это связано со слабым влиянием второго объекта на приёмники. 
%\input{tex/progDescription}
%\input{tex/conclusion}

\newpage

\addcontentsline{toc}{chapter}{Список литературы}
\renewcommand\bibname{СПИСОК ЛИТЕРАТУРЫ}

\begin{thebibliography}{00}
    \bibitem{1}
			Ю.Г. Соловейчик, М.Э. Рояк, М.Г. Персова Метод конечных элементов для скалярных и векторных задач Учеб. пособие. — Новосибирск: Изд-во НГТУ, 2007 — 896 с.
    \bibitem{2}
            А.Н. Тихонов, А.А. Самарский Уравнения математической физики: Учеб.пособие. / А.Н. Тихонов, А.А. Самарский — 6-е изд., — М: Изд-во МГУ, 1999 — 799 с.
\end{thebibliography}

\newpage
\chapter*{Приложение 3. Текст программы}
\addcontentsline{toc}{chapter}{Приложение 3. Текст программы}
\label{code: code}
\subsection*{Program.cs}
\lst{cs}{code/Program.cs}

\subsection*{LocalMatrix.cs}
\lst{cs}{code/LocalMatrix.cs}

\subsection*{LocalVector.cs}
\lst{cs}{code/LocalVector.cs}

\subsection*{MCG.cs}
\lst{cs}{code/MCG.cs}

\end{document}
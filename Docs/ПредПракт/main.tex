\documentclass[14pt,a4paper]{extreport}

\usepackage{style/style}
\usepackage{physics}
\usepackage{fancyhdr}

\fancypagestyle{plain}{%
\fancyhf{} % clear all header and footer fields
\fancyfoot[C]{\small\thepage}}
\renewcommand{\headrulewidth}{0pt}
\renewcommand{\footrulewidth}{0pt}
\pagestyle{plain}

\makeatletter
  \def\my@tag@font{\small}
  \def\maketag@@@#1{\hbox{\m@th\normalfont\my@tag@font#1}}
  \let\amsmath@eqref\eqref
  \renewcommand\eqref[1]{{\let\my@tag@font\relax\amsmath@eqref{#1}}}
\makeatother

\usepackage{titletoc}
\titlecontents{chapter}[0em]{\bfseries}{\thecontentslabel.\hspace{1em}}{}{\titlerule*[1pc]{.}\contentspage}
\titlecontents{section}[1.25em]{}{\thecontentslabel.\hspace{1em}}{}{\titlerule*[1pc]{.}\contentspage}
\titlecontents{subsection}[2.5em]{}{\thecontentslabel.\hspace{1em}}{}{\titlerule*[1pc]{.}\contentspage}

\begin{document}

% Отключение нумерации страниц
\pagenumbering{gobble}

%\chapter*{Аннотация}

Отчёт 100500 с., 4 ч., 100501 рис., 100502 табл., 100503 источника, 100504 прил.

МОДЕЛИРОВАНИЕ ЭЛЕКТРОМАГНИТНОГО ПОЛЯ, МЕТОД КОНЕЧНЫХ ЭЛЕМЕНТОВ, МНОГОЭТАПНАЯ СХЕМА РАЗДЕЛЕНИЯ ПОЛЕЙ

\textbf{Объект исследования:}  математическая модель электромагнитного поля в земной полости, создаваемое индукционным источником.
 
\textbf{Цель работы:} 
Сравнение результатов при использовании разделения поля на первичное и вторичное с результатами многоэтапной схемы разделения полей.

\textbf{Гипотеза исследования:} планомерное рассеивание поля в верних слоях земной коры. Сокращение вычислительных затрат при реализации многоэтапного разделения части поля.

\newpage
%\chapter*{Аннотация}

Отчёт 100500 с., 4 ч., 100501 рис., 100502 табл., 100503 источника, 100504 прил.

МОДЕЛИРОВАНИЕ ЭЛЕКТРОМАГНИТНОГО ПОЛЯ, МЕТОД КОНЕЧНЫХ ЭЛЕМЕНТОВ, МНОГОЭТАПНАЯ СХЕМА РАЗДЕЛЕНИЯ ПОЛЕЙ

\textbf{Объект исследования:}  математическая модель электромагнитного поля в земной полости, создаваемое индукционным источником.
 
\textbf{Цель работы:} 
Сравнение результатов при использовании разделения поля на первичное и вторичное с результатами многоэтапной схемы разделения полей.

\textbf{Гипотеза исследования:} планомерное рассеивание поля в верних слоях земной коры. Сокращение вычислительных затрат при реализации многоэтапного разделения части поля.
\tableofcontents
\newpage

% Включение нумерации страниц
\pagenumbering{arabic}
\setcounter{page}{3}
%\chapter*{Введение}

\addcontentsline{toc}{chapter}{Введение}

С увеличением потребности в природных ресурсах развивались способы поиска и исследования земных пород и руд. Наиболее распространёнными являются гравиразведка, магниторазведка, и электроразведка. Первые два можно считать естественными или природными, поскольку в их основе лежит использование гравитационного и магнитного полей Земли. При использовании электроразведки, поле создаётся искусственными источниками. В качестве данных источников можно рассматривать петлевые источники, вертикальные электрические линии (ВЭЛ) и самый новый из предложенных -- круговой электрический диполь (КЭД). Новизна его заключается в том, что это источник переменного поля, наземный аналог вертикальной электрической линии. 

Помимо возможности нахождения параметров среды, что является обратной задачей по определению, можно изучать и поведение самого электромагнитного поля. Зная его источник и параметры среды можно воспроизвести поле в пространстве земной коры, изучать характер его поведения в зависимости от количества неоднородностей в земной среде или параметров горизонтально-слоистой среды Земли.

При моделирования таких процессов, а особенно при истолковании сложных полей в которых было необходимо преобразовывать таким образом, чтобы разделить аномалии в зависимости от глубины расположения источников поля и обособлять такие изменения поля, которые соответствуют аномалиям тел простейших форм, обычные аналитические методы математического моделирования не помогут в силу больших временных затрат на расчёты. Вместо этого пользуются различными численными методами на различных ЭВМ. Для моделирования электромагнитных полей, лучшим вариантом можно считать метод конечных элементов (МКЭ). 

В данной работе будет рассмотренна возможность расчёта нестационарного поля КЭД в осесимметричной среде. Исследования проводились на воображаемой области размером $\Omega \in [0.001; 1000]_r \times [-1000; 0]_z$. Данная область включает в себя несколько разных слоев земных пород, имеющих различные удельные значения физических величин. Также рассмотрим процесс добавления нескольких аномальных пород, характерных различным земным рудам.

При написании работы использовались следующие языки программирования: для математических расчётов -- C$\#$ 12 на платформе .NET 8.0, для визуализации полученных результатов -- Python 3.12.2 с пакетами matplotlib версии 3.8.2 и numpy версии 1.26.4. 

\chapter{Постановка задачи}

\section{Аппарат математического моделирования}

Математическая модель, описывающая поведение электромагнитного поля в пространстве, известна в наши дни, как система уравнений Максвелла. Она позволяет описывать взаимосвязь сразу нескольких физических величин: напряжённости электрического $\overrightarrow{\textbf{E}}$ и магнитного $\overrightarrow{\textbf{H}}$ полей, а также индукцию магнитного поля $\overrightarrow{\textbf{B}}$. Большинство вычислительных задач электромагнетизма базируются на дифференциальной форме системы уравнений Максвелла (\ref{eq_1_1}) -- (\ref{eq_1_4}):

\begin{equation} \label{eq_1_1}
	\text{rot} \overrightarrow{\textbf{H}} = \overrightarrow{\textbf{J}^{\text{ст}}} + \sigma \overrightarrow{\textbf{E}} + \frac{\partial \left(\varepsilon \overrightarrow{\textbf{E}} \right)}{\partial t},
\end{equation}

\begin{equation} \label{eq_1_2}
	\text{rot} \overrightarrow{\textbf{E}} = - \frac{\partial \overrightarrow{\textbf{B}}}{\partial t},
\end{equation}

\begin{equation} \label{eq_1_3}
	\text{div} \overrightarrow{\textbf{B}} = 0,
\end{equation}

\begin{equation} \label{eq_1_4}
	\text{div} \varepsilon \overrightarrow{\textbf{E}} = \rho,
\end{equation}
где $\overrightarrow{\textbf{J}^{\text{ст}}}$ -- вектор плотностей сторонних токов, $\sigma$ -- удельная электрическая проводимость среды, $\varepsilon$ -- диэлектрическая проницаемость среды, а $\rho$ -- объёмная плотность стороннего электрического заряда.

Основное преимущество использования системы уравнений (\ref{eq_1_1}) -- (\ref{eq_1_4}) в дифференциальной форме, заключается в возможности учитывать нелинейность, анизотропию и другие нетривиальные аспекты среды расчётной области \cite{2}. 

Предлагаемый в работе подход к математическому конечноэлементному моделированию основан на технологии разделения полей, позволяющей существенно сократить вычислительные затраты. В рассматриваемой задаче под неоднородностями (аномалиями) будем понимать трёхмерные геологические объекты, отличные от сопротивления вмещающей горизонтально-слоистой среды.

Будем считать, что электромагнитное поле возбуждается круговым источником тока. В таком случае, в силу симметрии расчётной области будем решать задачу в цилиндрических координатах. Источник поля в таком случае описывается точкой, расположенной на некотором расстоянии, достаточно далёком от границы расчётной области. Тогда при условии однородности среды по магнитной проницаемости электромагнитное поле полностью описывается одной компонентой $A_{\varphi} = A_{\varphi}(r, z, t)$ вектор-потенциала $\overrightarrow{\textbf{A}}$. Функция $A_{\varphi}(r, z, t)$ может быть найдена из решения двумерного уравнения (\ref{eq_1_5}):

\begin{equation} \label{eq_1_5}
	-\frac{1}{\mu_0} \Delta A_{\varphi} + \frac{A_{\varphi}}{\mu_0 r^2} + \sigma \frac{\partial A_{\varphi}}{\partial t} = J_{\varphi},
\end{equation}
где $J_{\varphi}$ - источник стороннего тока, описываемый дельта-функцией, равной 1 в одной из подобласти, описывающей источник поля, и 0 во всех остальных. Удельную электропроводность $\sigma$ представим в виде кусочно-постоянной функции, описывающей физические характеристики горизонтально-слоистой среды. Потребуем, чтобы на всех границах было главное краевое условие $\left.A_{\varphi}(r, z, t)\right|_s = 0$. Тогда решение задачи (\ref{eq_1_5}) с главными однородными условиями на границах будем называть первичным или нормальным полем.

Решением задачи на оценку влияния аномальных объектов в горизонтально-слоистой среде будем называть вторичным (добавочным) полем. Также, как и в (\ref{eq_1_5}) потребуем на всех границах главное однородное краевое условие $\overrightarrow{\textbf{A}} \times \overrightarrow{\textbf{n}} |_s = 0$. Тогда, нестационарный процесс, возникающий после выключения источника тока в круглой обмотке, описывается следствием из уравнения (\ref{eq_1_1}):

\begin{equation} \label{eq_1_6}
	\text{rot} \left( \frac{1}{\mu_o} \text{rot} \overrightarrow{\textbf{A}}^{+} \right) + \sigma \frac{\partial \overrightarrow{\textbf{A}}^{+}}{\partial t} = (\sigma - \sigma_n) \overrightarrow{\textbf{E}}^n,
\end{equation}
где $\mu_0 = 4 \cdot \pi \cdot 10^{-7} = 1.25663753 \cdot 10^{-6}$ Гн/м, $\sigma_n$ -- значение удельной электрической проводимости среды на нормальном слое, $\overrightarrow{\textbf{E}}^n$ -- напряжённость первичного электрического поля, $\overrightarrow{\textbf{A}}^{+}$ -- значение вектор-потенциала на добавочном поле.

\section{Описание расчётной области}

Пусть у нас имеется расчётная область, геометрически представленная в виде параллелепипеда: $\Omega \in [-5500, 5500]_x \times [-5500, 5500]_y \times [-1500, 1500]_z$. Внутри неё имеются слои воздуха, глинозёма, песчаника, известняка и слюдяного сланца. Принимая во внимание факт того, что удельное электрическое сопротивление обратно пропорционально значению удельной электропроводности материала \cite{3}, тогда, зная значения сопротивления \cite{4}, найдём электропроводность интересующих материалов. Поскольку воздух является диэлектриком, значение удельной электропроводности для него $\sigma_{\text{возд.}} = 0$ См/м. Значения остальных слоев и аномальных областей \cite{5} приведены в таблице \ref{tab:fields}.

\begin{table}
	\caption{Удельные физические величины некоторых типов земных сред.}
	\centering
	\small
	\begin{tabularx}{1.0\textwidth}{| >{\raggedright\arraybackslash}X | >{\raggedright\arraybackslash}X | >{\raggedright\arraybackslash}X |}
		\hline
		\centering{Земная порода} & \centering{Удельное электрическое сопротивление Ом $\cdot$ м} & \centering{Удельная электропроводимость См / м}  \tabularnewline \hline
		
		\centering{Глинозём} & \centering{125}& \centering{0.008} \tabularnewline \hline
		
		\centering{Песчаник} & \centering{2000} & \centering{0.0005} \tabularnewline \hline
		
		\centering{Юрский известняк} & \centering{40} & \centering{0.025} \tabularnewline \hline
		
		\centering{Слюдяной сланец} & \centering{800} & \centering{0.00125} \tabularnewline \hline
		
		\centering{Осадочная порода} & \centering{0.1} &\centering{10} \tabularnewline \hline
		
		\centering{Грунтовые воды} & \centering{0.25} & \centering{4} \tabularnewline \hline
		
	\end{tabularx}
	\label{tab:fields}
\end{table}

Для решения задачи в максимально упрощённой, горизонтально-слоистой среде, зададим следующие значения где воздух находится на высоте $h_z = [0; 200]$, глинозём на глубине $h_z = [-9; 0]_z$, песчаник на $h_z = [-50; -9]$, известняк на $h_z = [-125; -50]$ и слюдяной сланец на $h_z = [-200; -125]$. В таком случае сама область в её двумерном случае изображена на рисунке \ref{fig:example}.

\begin{figure}
	\centering
	\includegraphics[width=1.0\linewidth]{images/"Figure_example".png}
	\caption{Расчётная область для двумерной задачи.}
	\label{fig:example}
\end{figure}
\chapter{Теоретическая часть}

\section{Многоэтапная схема разделения поля}

При решении многих векторных электромагнитных задач существенного повышения точности получаемого решения можно добиться в результате использования методов, основанных на разделении из искомого поля достаточно близкого к нему поля меньшей размерности. Сначала выделяется поле, создаваемое в среде, максимально упрощённой относительно исходной. Её решение берётся в качестве основного поля первого уровня. На базе этого поля формируется задача на добавочное поле, в которую включается часть неоднородностей исходной задачи, дающих максимальный вклад в искомое решение. Используемая для нахождения этого добавочного поля сетка строится так, чтобы максимально учесть влияние источников, порождённых включёнными в на этом этапе неоднородностями.

Далее в качестве основного поля будет учитываться сумма основного и добавочного на предыдущем этапе выделения. Новое добавочное поле будет формироваться из учёта следующих по влиянию на решение исходной задачи. Процесс можно продолжать до тех пор, пока не будут учтены все неоднородности среды.

Если рассматривать уравнение (\ref{eq_1_6}), то выходит, что излучаемое электромагнитное поле полностью описывается вектором-потенциалом $\overrightarrow{\textbf{A}}$, где значения магнитной индукции и электрической напряжённости определяются, как $\overrightarrow{\textbf{B}} = \text{rot} \overrightarrow{\textbf{A}}$ и $\overrightarrow{\textbf{E}} = -\frac{\partial \overrightarrow{\textbf{A}}}{\partial t}$ соответственно. Тогда каждое из полей $\overrightarrow{\textbf{A}}^{0}$ (нормальное) и $\overrightarrow{\textbf{A}}^{+}$ (добавочное) будет определять значения магнитной индукции и напряжённости электрического поля: $\overrightarrow{\textbf{B}}^0 = \text{rot} \overrightarrow{\textbf{A}}^0$ и $\overrightarrow{\textbf{E}}^0 = -\frac{\partial \overrightarrow{\textbf{A}}^0}{\partial t}$ для нормального, $\overrightarrow{\textbf{B}}^+ = \text{rot} \overrightarrow{\textbf{A}}^+$ и $\overrightarrow{\textbf{E}}^+ = -\frac{\partial \overrightarrow{\textbf{A}}^+}{\partial t}$ для добавочного. В нашем случае решение $\overrightarrow{\textbf{A}}^0$ получено из решения скалярной осесимметричной задачи (\ref{eq_1_5}).

Найдём вариационную постановку для уравнения ($\ref{eq_1_6}$):

\begin{equation} \label{eq_2_1}
	\frac{1}{\mu_0} \int \limits_{\Omega} \text{rot} \overrightarrow{\textbf{A}}^+ \text{rot} \overrightarrow{\Psi} d \Omega + \int \limits_{\Omega} \sigma \frac{\partial \overrightarrow{\textbf{A}}^+}{\partial t} \overrightarrow{\Psi} d \Omega = \int \limits_{\Omega}(\sigma - \sigma_n) \overrightarrow{\textbf{E}^0} \overrightarrow{\Psi} d \Omega.
\end{equation}

Из (\ref{eq_2_1}) получим матричное уравнение для добавочного поля:

\begin{equation} \label{eq_2_2}
	\left(\frac{1}{\mu_0} \hat{\textbf{G}} + \sigma \frac{1}{\Delta t} \hat{\textbf{M}} \right) \text{q}^i = (\sigma - \sigma_n) \overrightarrow{\textbf{E}}^0 \hat{\textbf{M}} + \sigma \frac{1}{\Delta t} \hat{\textbf{M}}\text{q}^{i-1}.
\end{equation}
%\chapter{Практическая часть}

\section{Формат входных и выходных данных}

Входные данные содержатся в папке "Data/Input/". Файл "WholeMesh.txt" содержит данные о трёхмерной сетке, из которой автоматически строится сетка для решения двумерной задачи на нормальном поле. В файле содержится информация о границах расчётной  области по $x$, $y$, $z$, количество необходимых разбиений для каждой оси, коэффициенты разрядки, количество областей с разными значениями удельной электропроводности и информация о границах расчётной области. Полностью формат изображен на рисунке \ref{fig:TextWholeMesh}.

\begin{figure}
	\centering
	\includegraphics[width=0.8\linewidth]{images/"inputMeshText".png}
	\caption{Входной формат сетки по пространству}
	\label{fig:TextWholeMesh}
\end{figure}

Входные данные для учёта поля влияния хранятся в папке "Data/Input/Anomalies/". Каждый файл, находящийся в этой папке, содержит примерно похожий формат хранения, как и для основной сетки. Задаются границы по осям $x$, $y$, $z$, количество необходимых разбиений для каждой оси, коэффициенты разрядки, значения удельной электропроводности на аномальной области и границы этой области. Полностью формат изображен на рисунке \ref{fig:TextAnomalyMesh}.

\begin{figure}
	\centering
	\includegraphics[width=0.8\linewidth]{images/"inputAnomalyText".png}
	\caption{Входной формат сетки по пространству для аномалии}
	\label{fig:TextAnomalyMesh}
\end{figure}

Входные данные для сетки по времени, содержатся в файле "Time.txt", в папке "Data/Input/" и содержат четыре значения: время начала и конца, количество разбиений и коэффициент разрядки.

\section{Сборка глобальной матрицы и глобального вектора правой части}

При формировании матрицы \textbf{A} для решения СЛАУ необходимо учитывать соответствие локальной к глобальной нумерации каждого узла. Глобальная нумерация узлов сетки однозначно определяет вклад локальной матрицы в соответствующие строчки и столбцы матрицы \textbf{A}. Поэтому, зная глобальную нумерацию узлов конечного элемента, можно определить какие элементы глобальной матрицы изменятся при добавлении в нее локальной. Аналогичным образом определяется вклад локального вектора правой части в глобальный.

\lst{cs}{code/AddLocalMatrix.cs}

\section{Учёт краевых условий}

Поскольку в решаемой задаче у нас на всех границах задаётся однородное краевое условие первого рода, технически необходимо в соответствующей строчке матрицы обнулить вне диагональные элементы, на диагонали поставить значение 1, а в соответствующую строчку вектора правой части поставить значение краевого условия на этой границе, т.е. в нашем случае тоже обнулить.

\section{Решение СЛАУ}

Для решения СЛАУ мы будем использовать локально-оптимальную схему \textbf{(очень хочется починить LU предобусловливание)}. Это хороший метод решения систем уравнений для несимметричных матриц. Перед решением СЛАУ задаются параметры для досрочного выхода из итерационного процесса, а именно: выход по максимальному количеству совершённых итераций и минимальному значению нормы вектора невязки.

\lst{cs}{code/LOS.cs}

В результате решения СЛАУ мы получим вектор $q$ весов базисных функций, на которые раскладывается функция $A_{\varphi}^0$ или вектор-функция $\overrightarrow{\textbf{A}}^{+}$. Учитывая построение базисных функций, компонентами этого вектора будут значения функции в соответствующих узлах сетки.

\section{Определение значения вектор-потенциала и напряжённости электрического поля}

После решения СЛАУ вида (\ref{eq_2_27}) или (\ref{eq_2_43}), необходимо найти напряжённость электрического поля по формуле (\ref{eq_2_28}). Пользуясь аналитическим представлением из (\ref{eq_2_23}), программная реализация будет выглядеть следующим образом.

\lst{cs}{code/E_generator.cs}

Поскольку в качестве конечных элементов использовались прямоугольники для двумерной и прямые параллелепипеды для трёхмерной задач, то можно упростить алгоритм нахождения значения функции на элементе. Можно не перебирать каждый элемент отдельно и проверять значение интересующей точки на принадлежность ему, а последовательно сравнивать координаты точки со значениями на разбиениях по осям координат. Тогда сложность алгоритма будет не $O(n^2)$ для двумерной или $O(n^3)$ для трёхмерной задач, а $O(k \cdot n)$, где $n$ -- количество отрезков, на которые разбиваются оси координат.

\lst{cs}{code/Checker.cs}
 
\section{Проверка полученных результатов}

Проверку полученных результатов решения СЛАУ будем из закона индукции Фарадея (\ref{eq_1_2}) и теоремы о циркуляции магнитного поля (\ref{eq_1_1}). Учитывая (\ref{eq_2_27}) -- (\ref{eq_2_29}), получим выражение для $\overrightarrow{\textbf{B}}$:

\begin{equation} \label{eq_3_1}
	\overrightarrow{\textbf{B}} = \text{rot} \overrightarrow{\textbf{A}} = 
	\begin{vmatrix}
		\textbf{i} & \textbf{j} & \textbf{k}\\
		\frac{\partial}{\partial x} & \frac{\partial}{\partial y} & \frac{\partial}{\partial z}\\
		A_x & A_y & 0
	\end{vmatrix}
	= -\frac{\partial A_y}{\partial z} \textbf{i} + \frac{\partial A_x}{\partial z} \textbf{j} + \left(\frac{\partial A_y}{\partial x} - \frac{\partial A_x}{\partial y}\right) \textbf{k}.
\end{equation}

Исходя из теории конечно-разностных схем \cite{7}, численно определим значения для частных первых производных в выражении (\ref{eq_3_1}):

\begin{equation} \label{eq_3_2}
	\frac{\partial A_{x_i}}{\partial x_k} + o\left(h_{x_k}^3\right) = \frac{A_{x_i}^{j+1} - A_{x_i}^{j-1}}{2h_{x_k}},
\end{equation}
где $x_k$ -- переменная по которой проводится дифференцирование, $x_i$ -- соответствующая компонента вектора-потенциала $\overrightarrow{\textbf{A}}$, $A_{x_i}^{j+1} = A_{x_i}\left(x_{0i} + h_{x_k}\right)$, $A_{x_i}^{j-1} = A_{x_{i}}\left(x_{0i} - h_{x_k}\right)$, $h_{x_k}$ -- шаг от точки, в которой необходимо найти значение производной, равный $10^{-10}$.

\textbf{Здесь будет программная реализация $\downarrow$}


\chapter{Исследования}

\section{Исследование первичного поля}

Пусть источник индукционного поля лежит на расстоянии $R = 500$ м от оси симметрии и имееет силу тока, равную $J_{\varphi} = 1.0$ А. Также условимся, что источник работал достаточно долго, чтобы создать стабильное электромагнитное поле. Сетка по времени равномерная: $t=[1.0; 1.05]$ на 200 временных слоёв. После истечения первой секунды мы отключим наш источник, т.е. $J_{\varphi} = 0.0$ A при $t > 1.0$. На рисунках \ref{fig:A_phi_0} -- \ref{fig:E_phi_2} представлено распространение этого поля в среде в начальный, промежуточных и последний момент времени.

\begin{figure}
	\centering
	\includegraphics[width=1.0\linewidth]{images/Answer_A_time_layer_1.png}
	\caption{Решение $A_{\varphi}$ при $t = 1.0с$}
	\label{fig:A_phi_0}
\end{figure}

\begin{figure}
	\centering
	\includegraphics[width=1.0\linewidth]{images/Answer_E_time_layer_1.png}
	\caption{Решение $E_{\varphi}$ при $t = 1.0с$}
	\label{fig:E_phi_0}
\end{figure}

\begin{figure}
	\centering
	\includegraphics[width=1.0\linewidth]{images/Answer_A_time_layer_1.0250000000000083.png}
	\caption{Решение $A_{\varphi}$ при $t = 1.025с$}
	\label{fig:A_phi_1}
\end{figure}

\begin{figure}
	\centering
	\includegraphics[width=1.0\linewidth]{images/Answer_E_time_layer_1.0250000000000083.png}
	\caption{Решение $E_{\varphi}$ при $t = 1.025с$}
	\label{fig:E_phi_1}
\end{figure} 

\begin{figure}
	\centering
	\includegraphics[width=1.0\linewidth]{images/Answer_A_time_layer_1.05.png}
	\caption{Решение $A_{\varphi}$ при $t = 1.05с$}
	\label{fig:A_phi_2}
\end{figure}

\begin{figure}
	\centering
	\includegraphics[width=1.0\linewidth]{images/Answer_E_time_layer_1.05.png}
	\caption{Решение $A_{\varphi}$ при $t = 1.05с$}
	\label{fig:E_phi_2}
\end{figure} 

Расположим на расчётной области приёмники в каждой горизонтально-слоистой среде и проведем замеры значений вектор-потенциала и электрического поля в точках $(2500; 0; -100)$, $(2500; 0; -200)$, $(10; 0; -700)$, $(1000; 0; -1250)$. Отобразим на графиках \ref{fig:NatA} -- \ref{fig:LogE} полученные значения.

\begin{figure}
	\centering
	\includegraphics[width=0.8\linewidth]{images/Normal_A.png}
	\caption{Зависимость значения $A_{\varphi}$ от времени в разных приёмниках}
	\label{fig:NatA}
\end{figure}

\begin{figure}
	\centering
	\includegraphics[width=0.8\linewidth]{images/Normal_E.png}
	\caption{Зависимость значения $E_{\varphi}$ от времени в разных приёмниках}
	\label{fig:NatE}
\end{figure}


\begin{figure}
	\centering
	\includegraphics[width=0.8\linewidth]{images/Log_A.png}
	\caption{Зависимость значения $A_{\varphi}$ от времени в разных приёмниках (логарифмическая шкала по оси абсцисс)}
	\label{fig:LogA}
\end{figure}

\begin{figure}
	\centering
	\includegraphics[width=0.8\linewidth]{images/Log_E.png}
	\caption{Зависимость значения $E_{\varphi}$ от времени в разных приёмниках (логарифмическая шкала по оси абсцисс)}
	\label{fig:LogE}
\end{figure} 

Как видим, значения вектор-потенциала и электрической напряжённости поля не имеют каких-либо резких колебаний. Из этого можно заключить, что, как и предполагалось, никаких аномальных зон в исследуемой области нет. 

\section{Исследование при разделении нормального и добавочного поля}

Добавим в нашу область аномальный объект со следующими границами: $[-5500; 5500]_x \times [2205; 2355]_y \times [-180; -80]_z$ и значением $\sigma = 4$. Будем искать решение из уравнения на добавочное поле (\ref{eq_1_6}). Получим решения изображенные на рисунках \ref{fig:A_plus_t0} -- \ref{fig:E_plus_t2}.

\begin{figure}
	\centering
	\includegraphics[width=1.0\linewidth]{images/Answer_A_plus_time_layer_1.png}
	\caption{Решение $\overrightarrow{\textbf{A}}^+$ при $t = 1.0с$}
	\label{fig:A_plus_t0}
\end{figure} 


\begin{figure}
	\centering
	\includegraphics[width=1.0\linewidth]{images/Answer_E_plus_time_layer_1.png}
	\caption{Решение $\overrightarrow{\textbf{E}}^+$ при $t = 1.0с$}
	\label{fig:E_plus_t0}
\end{figure} 


\begin{figure}
	\centering
	\includegraphics[width=1.0\linewidth]{images/Answer_A_plus_time_layer_1.0250000000000006.png}
	\caption{Решение $\overrightarrow{\textbf{A}}^+$ при $t = 1.025с$}
	\label{fig:A_plus_t1}
\end{figure} 


\begin{figure}
	\centering
	\includegraphics[width=1.0\linewidth]{images/Answer_E_plus_time_layer_1.0250000000000006.png}
	\caption{Решение $\overrightarrow{\textbf{E}}^+$ при $t = 1.025с$}
	\label{fig:E_plus_t1}
\end{figure} 

\begin{figure}
	\centering
	\includegraphics[width=1.0\linewidth]{images/Answer_A_plus_time_layer_1.05.png}
	\caption{Решение $\overrightarrow{\textbf{A}}^+$ при $t = 1.05с$}
	\label{fig:A_plus_t2}
\end{figure} 


\begin{figure}
	\centering
	\includegraphics[width=1.0\linewidth]{images/Answer_E_plus_time_layer_1.05.png}
	\caption{Решение $\overrightarrow{\textbf{E}}^+$ при $t = 1.05с$}
	\label{fig:E_plus_t2}
\end{figure} 

Рассмотрим для $t = 1.0, t = 1.025, t = 1.05$ значения $\overrightarrow{\textbf{A}}^+$ и $\overrightarrow{\textbf{E}}^+$ на линии, перпендикулярно проходящей к аномальному объекту по оси $y$ при $x = 0.0, z = -130.0$. Получим следующее:

\begin{figure}
	\centering
	\includegraphics[width=0.5\linewidth]{images/Normal_A(y)_1.png}
	\caption{Решение $\overrightarrow{\textbf{A}}^+$ на линии $(0.0, y, -130.0)$ при $t = 1.0с$}
	\label{fig:A_line_t0}
\end{figure} 

\begin{figure}
	\centering
	\includegraphics[width=0.5\linewidth]{images/Normal_E(y)_1.png}
	\caption{Решение $\overrightarrow{\textbf{E}}^+$ на линии $(0.0, y, -130.0)$ при $t = 1.0с$}
	\label{fig:E_line_t0}
\end{figure} 

\begin{figure}
	\centering
	\includegraphics[width=0.5\linewidth]{images/Normal_A(y)_2.png}
	\caption{Решение $\overrightarrow{\textbf{A}}^+$ на линии $(0.0, y, -130.0)$ при $t = 1.025с$}
	\label{fig:A_line_t1}
\end{figure} 

\begin{figure}
	\centering
	\includegraphics[width=0.5\linewidth]{images/Normal_E(y)_2.png}
	\caption{Решение $\overrightarrow{\textbf{E}}^+$ на линии $(0.0, y, -130.0)$  при $t = 1.025с$}
	\label{fig:E_line_t1}
\end{figure} 

\begin{figure}
	\centering
	\includegraphics[width=0.5\linewidth]{images/Normal_A(y)_3.png}
	\caption{Решение $\overrightarrow{\textbf{A}}^+$ на линии $(0.0, y, -130.0)$  при $t = 1.05с$}
	\label{fig:A_line_t2}
\end{figure} 

\begin{figure}
	\centering
	\includegraphics[width=0.5\linewidth]{images/Normal_A(y)_3.png}
	\caption{Решение $\overrightarrow{\textbf{A}}^+$ на линии $(0.0, y, -130.0)$  при $t = 1.05с$}
	\label{fig:E_line_t2}
\end{figure} 


Суммируем полученный результат с нормальным полем и получим состояние поля в разные моменты времени, изображённые на рисунках \ref{fig:A_Istage_t0} -- \ref{fig:E_Istage_t2}.

\begin{figure}
	\centering
	\includegraphics[width=1.0\linewidth]{images/Answer_A_Istage_time_layer_1.png}
	\caption{Решение суммарного поля $\overrightarrow{\textbf{A}}$ при $t = 1.0с$}
	\label{fig:A_Istage_t0}
\end{figure} 


\begin{figure}
	\centering
	\includegraphics[width=1.0\linewidth]{images/Answer_E_Istage_time_layer_1.png}
	\caption{Решение суммарного поля $\overrightarrow{\textbf{E}}$ при $t = 1.0с$}
	\label{fig:E_Istage_t0}
\end{figure} 


\begin{figure}
	\centering
	\includegraphics[width=1.0\linewidth]{images/Answer_A_Istage_time_layer_1.0250000000000006.png}
	\caption{Решение суммарного поля $\overrightarrow{\textbf{A}}$ при $t = 1.025с$}
	\label{fig:A_Istage_t1}
\end{figure} 


\begin{figure}
	\centering
	\includegraphics[width=1.0\linewidth]{images/Answer_E_Istage_time_layer_1.0250000000000006.png}
	\caption{Решение суммарного поля $\overrightarrow{\textbf{E}}$ при $t = 1.025с$}
	\label{fig:E_Istage_t1}
\end{figure} 

\begin{figure}
	\centering
	\includegraphics[width=1.0\linewidth]{images/Answer_A_Istage_time_layer_1.05.png}
	\caption{Решение суммарного поля $\overrightarrow{\textbf{A}}$ при $t = 1.05с$}
	\label{fig:A_Istage_t2}
\end{figure} 


\begin{figure}
	\centering
	\includegraphics[width=1.0\linewidth]{images/Answer_E_Istage_time_layer_1.05.png}
	\caption{Решение суммарного поля $\overrightarrow{\textbf{E}}$ при $t = 1.05с$}
	\label{fig:E_Istage_t2}
\end{figure} 

Полученные значения \ref{fig:A_Log_added} -- \ref{fig:E_Log_added} $\overrightarrow{\textbf{A}}$ и $\overrightarrow{\textbf{E}}$ рассмотрим на приёмниках.

\begin{figure}
	\centering
	\includegraphics[width=0.8\linewidth]{images/Log_A_obj1.png}
	\caption{Решение суммарного поля $\overrightarrow{\textbf{E}}$ при $t = 1.05с$}
	\label{fig:A_Log_added}
\end{figure} 


\begin{figure}
	\centering
	\includegraphics[width=0.8\linewidth]{images/Log_E_obj1.png}
	\caption{Решение суммарного поля $\overrightarrow{\textbf{E}}$ при $t = 1.05с$}
	\label{fig:E_Log_added}
\end{figure} 

Сравнивая показатели на приёмниках до добавления аномалии \ref{fig:LogA} -- \ref{fig:LogE} и после \ref{fig:A_Log_added} -- \ref{fig:A_Log_added}, можно заметить, что значения напряжённости электрического поля на красном приёмнике претерпели наиболее сильные изменения, т.к. он стал более похожим на гиперболу, нежели прямую линию. Значения на синем приёмнике начали изменяться уже в последние сотые секунды исследования. 

\section{Исследование многоэтапного разделения нормального и добавочных полей}

Добавим ещё один аномальный объект со следующими границами: $[-1305; -1050]_x \times [-2255; -2178]_y \times [-1250; -800]_z$ и значением $\sigma = 17$. Будем искать решение из уравнения на добавочное поле (\ref{eq_1_6}). Получим решения в разные моменты времени изображенные на рисунках \ref{fig:A_2plus_t0} -- \ref{fig:E_2plus_t2}.

\begin{figure}
	\centering
	\includegraphics[width=1.0\linewidth]{images/Answer_A_2plus_time_layer_1.png}
	\caption{Решение $\overrightarrow{\textbf{A}}^+$ при $t = 1.0с$}
	\label{fig:A_2plus_t0}
\end{figure} 


\begin{figure}
	\centering
	\includegraphics[width=1.0\linewidth]{images/Answer_E_2plus_time_layer_1.png}
	\caption{Решение $\overrightarrow{\textbf{E}}^+$ при $t = 1.0с$}
	\label{fig:E_2plus_t0}
\end{figure} 


\begin{figure}
	\centering
	\includegraphics[width=1.0\linewidth]{images/Answer_A_2plus_time_layer_1.0250000000000006.png}
	\caption{Решение $\overrightarrow{\textbf{A}}^+$ при $t = 1.025с$}
	\label{fig:A_2plus_t1}
\end{figure} 


\begin{figure}
	\centering
	\includegraphics[width=1.0\linewidth]{images/Answer_E_2plus_time_layer_1.0250000000000006.png}
	\caption{Решение $\overrightarrow{\textbf{E}}^+$ при $t = 1.025с$}
	\label{fig:E_2plus_t1}
\end{figure} 

\begin{figure}
	\centering
	\includegraphics[width=1.0\linewidth]{images/Answer_A_2plus_time_layer_1.05.png}
	\caption{Решение $\overrightarrow{\textbf{A}}^+$ при $t = 1.05с$}
	\label{fig:A_2plus_t2}
\end{figure} 


\begin{figure}
	\centering
	\includegraphics[width=1.0\linewidth]{images/Answer_E_2plus_time_layer_1.05.png}
	\caption{Решение $\overrightarrow{\textbf{E}}^+$ при $t = 1.05с$}
	\label{fig:E_2plus_t2}
\end{figure} 

Рассмотрим для $t = 1.0, t = 1.025, t = 1.05$ значения $\overrightarrow{\textbf{A}}^+$ и $\overrightarrow{\textbf{E}}^+$ на линии, перпендикулярно проходящей к аномальному объекту по оси $y$ при $x = -1177.5, z = -1050.0$. Получим следующее:

\begin{figure}
	\centering
	\includegraphics[width=0.5\linewidth]{images/Normal_A_obj2_1.png}
	\caption{Решение $\overrightarrow{\textbf{A}}$ на линии $(0.0, y, -130.0)$ при $t = 1.025с$}
	\label{fig:A_2line_t0}
\end{figure} 

\begin{figure}
	\centering
	\includegraphics[width=0.5\linewidth]{images/Normal_E_obj2_1.png}
	\caption{Решение $\overrightarrow{\textbf{E}}$ на линии $(0.0, y, -130.0)$ при $t = 1.025с$}
	\label{fig:E_2line_t0}
\end{figure} 

\begin{figure}
	\centering
	\includegraphics[width=0.5\linewidth]{images/Normal_A_obj2_2.png}
	\caption{Решение $\overrightarrow{\textbf{A}}$ на линии $(0.0, y, -130.0)$ при $t = 1.025с$}
	\label{fig:A_2line_t1}
\end{figure} 

\begin{figure}
	\centering
	\includegraphics[width=0.5\linewidth]{images/Normal_E_obj2_2.png}
	\caption{Решение $\overrightarrow{\textbf{E}}$ на линии $(0.0, y, -130.0)$  при $t = 1.025с$}
	\label{fig:E_2line_t1}
\end{figure} 

\begin{figure}
	\centering
	\includegraphics[width=0.5\linewidth]{images/Normal_A_obj2_3.png}
	\caption{Решение $\overrightarrow{\textbf{A}}$ на линии $(0.0, y, -130.0)$  при $t = 1.05с$}
	\label{fig:A_2line_t2}
\end{figure} 

\begin{figure}
	\centering
	\includegraphics[width=0.5\linewidth]{images/Normal_E_obj2_2.png}
	\caption{Решение $\overrightarrow{\textbf{A}}$ на линии $(0.0, y, -130.0)$  при $t = 1.05с$}
	\label{fig:E_2line_t2}
\end{figure} 

Суммируем полученный результат с полем без аномалий и получим состояние поля в разные моменты времени, изображённые на рисунках \ref{fig:A_IIstage_t0} -- \ref{fig:E_IIstage_t2}.

\begin{figure}
	\centering
	\includegraphics[width=1.0\linewidth]{images/Answer_A_IIstage_time_layer_1.png}
	\caption{Решение суммарного поля $\overrightarrow{\textbf{A}}$ при $t = 1.0с$}
	\label{fig:A_IIstage_t0}
\end{figure} 


\begin{figure}
	\centering
	\includegraphics[width=1.0\linewidth]{images/Answer_E_IIstage_time_layer_1.png}
	\caption{Решение суммарного поля $\overrightarrow{\textbf{E}}$ при $t = 1.0с$}
	\label{fig:E_IIstage_t0}
\end{figure} 


\begin{figure}
	\centering
	\includegraphics[width=1.0\linewidth]{images/Answer_A_IIstage_time_layer_1.0250000000000006.png}
	\caption{Решение суммарного поля $\overrightarrow{\textbf{A}}$ при $t = 1.025с$}
	\label{fig:A_IIstage_t1}
\end{figure} 


\begin{figure}
	\centering
	\includegraphics[width=1.0\linewidth]{images/Answer_E_IIstage_time_layer_1.0250000000000006.png}
	\caption{Решение суммарного поля $\overrightarrow{\textbf{E}}$ при $t = 1.025с$}
	\label{fig:E_IIstage_t1}
\end{figure} 

\begin{figure}
	\centering
	\includegraphics[width=1.0\linewidth]{images/Answer_A_IIstage_time_layer_1.05.png}
	\caption{Решение суммарного поля $\overrightarrow{\textbf{A}}$ при $t = 1.05с$}
	\label{fig:A_IIstage_t2}
\end{figure} 


\begin{figure}
	\centering
	\includegraphics[width=1.0\linewidth]{images/Answer_E_IIstage_time_layer_1.05.png}
	\caption{Решение суммарного поля $\overrightarrow{\textbf{E}}$ при $t = 1.05с$}
	\label{fig:E_IIstage_t2}
\end{figure} 

Полученные значения \ref{fig:A_Log_added2} -- \ref{fig:E_Log_added2} $\overrightarrow{\textbf{A}}$ и $\overrightarrow{\textbf{E}}$ рассмотрим на приёмниках.

\begin{figure}
	\centering
	\includegraphics[width=0.8\linewidth]{images/Log_A_obj2.png}
	\caption{Решение суммарного поля $\overrightarrow{\textbf{E}}$ при $t = 1.05с$}
	\label{fig:A_Log_added2}
\end{figure} 


\begin{figure}
	\centering
	\includegraphics[width=0.8\linewidth]{images/Log_E_obj2.png}
	\caption{Решение суммарного поля $\overrightarrow{\textbf{E}}$ при $t = 1.05с$}
	\label{fig:E_Log_added2}
\end{figure} 

Сравнивая показатели на приёмниках до добавления аномалии \ref{fig:LogA} -- \ref{fig:LogE} и после \ref{fig:A_Log_added2} -- \ref{fig:A_Log_added2}, можно заметить, что значения напряжённости электрического поля ни на одном из приёмников не претерпели изменения, по всей видимости из-за неудачного их расположения. Для изменения ситуации необходимо было бы увеличить размеры расчётной области по пространству и временного диапазона. 

Рассмотрим теперь поле с двумя объектами сразу. Для этого сначала добавим первый объект к чистой расчётной области, после используя его в качестве нормального, добавим вторую аномалию.

\begin{figure}
	\centering
	\includegraphics[width=1.0\linewidth]{images/Answer_A_both_time_layer_1.png}
	\caption{Решение суммарного поля $\overrightarrow{\textbf{A}}$ при $t = 1.0с$}
	\label{fig:A_both_t0}
\end{figure} 


\begin{figure}
	\centering
	\includegraphics[width=1.0\linewidth]{images/Answer_E_both_time_layer_1.png}
	\caption{Решение суммарного поля $\overrightarrow{\textbf{E}}$ при $t = 1.0с$}
	\label{fig:E_both_t0}
\end{figure} 


\begin{figure}
	\centering
	\includegraphics[width=1.0\linewidth]{images/Answer_A_both_time_layer_1.0250000000000006.png}
	\caption{Решение суммарного поля $\overrightarrow{\textbf{A}}$ при $t = 1.025с$}
	\label{fig:A_both_t1}
\end{figure} 


\begin{figure}
	\centering
	\includegraphics[width=1.0\linewidth]{images/Answer_E_both_time_layer_1.0250000000000006.png}
	\caption{Решение суммарного поля $\overrightarrow{\textbf{E}}$ при $t = 1.025с$}
	\label{fig:E_both_t1}
\end{figure} 

\begin{figure}
	\centering
	\includegraphics[width=1.0\linewidth]{images/Answer_A_both_time_layer_1.05.png}
	\caption{Решение суммарного поля $\overrightarrow{\textbf{A}}$ при $t = 1.05с$}
	\label{fig:A_both_t2}
\end{figure} 


\begin{figure}
	\centering
	\includegraphics[width=1.0\linewidth]{images/Answer_E_both_time_layer_1.05.png}
	\caption{Решение суммарного поля $\overrightarrow{\textbf{E}}$ при $t = 1.05с$}
	\label{fig:E_both_t2}
\end{figure} 

Полученные значения \ref{fig:A_Log_both} -- \ref{fig:E_Log_both} $\overrightarrow{\textbf{A}}$ и $\overrightarrow{\textbf{E}}$ рассмотрим на приёмниках.

\begin{figure}
	\centering
	\includegraphics[width=0.8\linewidth]{images/Log_A_both.png}
	\caption{Решение суммарного поля $\overrightarrow{\textbf{E}}$ при $t = 1.05с$}
	\label{fig:A_Log_both}
\end{figure} 


\begin{figure}
	\centering
	\includegraphics[width=0.8\linewidth]{images/Log_E_both.png}
	\caption{Решение суммарного поля $\overrightarrow{\textbf{E}}$ при $t = 1.05с$}
	\label{fig:E_Log_both}
\end{figure} 

Сравнивая показатели на приёмниках до добавления обеих аномалий \ref{fig:LogA} -- \ref{fig:LogE} и после \ref{fig:A_Log_both} -- \ref{fig:E_Log_both}, можно заметить, что значения напряжённости электрического поля на красном приёмнике все те же изменения, что и без учёта второго объекта. Это связано со слабым влиянием второго объекта на приёмники. 
%\input{tex/progDescription}
%\input{tex/conclusion}

\newpage

\addcontentsline{toc}{chapter}{Список используемой литературы}
\renewcommand\bibname{СПИСОК ИСПОЛЬЗУЕМой ЛИТЕРАТУРЫ}

\begin{thebibliography}{00}
    \bibitem{1}
			А.А. Логачев, В.П. Захаров Магниторазведка. - 5 изд. - Ленинград: Недра, 1979. - 350 с.
    \bibitem{2}
    		Pavel Solin Partial Differential Equations and the Finite Element Method. - Hoboken, New Jersey: A JOHN WILEY $\&$ SONS, INC., 2006.
   	\bibitem{3}		
   			Кухлинг Х. Справочник по физике. Пер. с нем., М.: Мир, 1982, стр. 475 (табл. 39)
    \bibitem{4}
			Ю.Г. Соловейчик, М.Э. Рояк, М.Г. Персова Метод конечных элементов для скалярных и векторных задач Учеб. пособие. — Новосибирск: Изд-во НГТУ, 2007 — 896 с.
	\bibitem{5}
    		М.Г. Персова, Ю.Г. Соловейчик, Д.В. Вагин, П.А. Домников, Ю.И. Кошкина Численные методы в уравнениях математической физики.  - Новосибирск: Изд-во НГТУ, 2016 — 60 с.
	\bibitem{6}
			А.Н. Тихонов, А.А. Самарский Уравнения математической физики: Учеб.пособие. / А.Н. Тихонов, А.А. Самарский — 6-е изд., — М: Изд-во МГУ, 1999 — 799 с.
  	\bibitem{7}
  			М.Ю.Баландин, Э.П.Шурина Методы решения СЛАУ большой размерности: Учеб. пособие. - Новосибирск: Изд-во НГТУ, 2000 — 70 с.
  	\bibitem{8}
  			М.Ю.Баландин, Э.П.Шурина Векторный метод конечных элементов: Учеб. пособие. - Новосибирск: Изд-во НГТУ, 2001 — 69 с.
  	\bibitem{9}
  			М. Г. Персова, Ю. Г. Соловейчик, Г. М. Тригубович, М. В. Абрамов, А. А. Заборцева О вычислении трёхмерного нестационарного поля вертикальной электрической линии в удалённой обсаженной скважине // Сибирский журнал индустриальной математики. - 2007. - №3(31). - С. 114 - 127.
  	\bibitem{10}		
  			М.Г. Персова, Ю.Г. Соловейчик, М.Г. Токарева, М.В. Абрамов 3D-моделирование процессов индукционной вызванной поляризации при возбуждении токовой петлей и проблема эквивалентности // Научный вестник НГТУ. - 2013. - №2(51). - С. 53 - 61.
\end{thebibliography}

\newpage
%\hypertarget{a1}{}
\chapter*{Приложение 1. Тестирование двумерной задачи на полиномах}
\addcontentsline{toc}{chapter}{Приложение 1. Тестирование двумерной задачи на полиномах}

Проведем сначала тестирование программы на работоспособность для уравнения (\ref{eq_4_1}). Образец расчетной области изображен на рисунке \ref{fig:exampleOf3DMesh}. Это область $\Omega = [1.0, 2.0]_r \times [1.0, 2.0]_z$, она содержит 16 узлов, а на всех границах будем задавать первые краевые условия.

\begin{equation} \label{eq_4_1}
	-\frac{1}{\mu_0 r} \frac{\partial}{\partial r} \left(r \frac{\partial u}{\partial r}\right) - \frac{1}{\mu_0} \frac{\partial^2 u}{\partial z^2} + \frac{u}{\mu_0 r^2} + \sigma \frac{\partial A_{\varphi}}{\partial t} = f,
\end{equation}

\begin{figure}
	\centering
	\includegraphics[width=0.75\linewidth]{images/"TestMesh".png}
	\caption{Расчетная область}
	\label{fig:exampleOf3DMesh}
\end{figure}

\begin{table}
	\caption{Тестирование при $u = 2$, $f = \frac{2}{r^2}$, $\mu_0 = 1,\sigma = 0$}
	\centering
	\small
	\begin{tabularx}{1.0\textwidth}{| >{\raggedright\arraybackslash}X | >{\raggedright\arraybackslash}X | >{\raggedright\arraybackslash}X |>{\raggedright\arraybackslash}X |}
		\hline
		\centering{Узел} & \centering{Значение} & \centering{Абсолютная погрешность} & \centering{Относительная погрешность} \tabularnewline \hline
		
		
		
		\centering{(${}^4/_3$; ${}^4/_3$)} & \centering{2.00226896E+000}& \centering{2.26896083E-003} & \centering{1.13448042E-003} \tabularnewline \hline
		
		\centering{(${}^5/_3$; ${}^4/_3$)} & \centering{2.00130487E+000} & \centering{1.30486533E-003} & \centering{6.52432666E-004} \tabularnewline \hline
		
		\centering{(${}^4/_3$; ${}^5/_3$)} & \centering{2.00226896E+000} & \centering{2.26896083E-003} & \centering{1.13448042E-003} \tabularnewline \hline
		
		\centering{(${}^5/_3$; ${}^5/_3$)} & \centering{2.00130487E+000} & \centering{1.30486533E-003} & \centering{6.52432666E-004} \tabularnewline \hline
		
	\end{tabularx}
	\label{tab:test1}
\end{table}

\begin{table}
	\caption{Тестирование при $u = r$, $f = 0$, $\mu_0 = 1,\sigma = 0$}
	\centering
	\small
	\begin{tabularx}{1.0\textwidth}{| >{\raggedright\arraybackslash}X | >{\raggedright\arraybackslash}X | >{\raggedright\arraybackslash}X |>{\raggedright\arraybackslash}X |}
		\hline
		\centering{Узел} & \centering{Значение} & \centering{Абсолютная погрешность} & \centering{Относительная погрешность} \tabularnewline \hline
		
		
		
		\centering{(${}^4/_3$; ${}^4/_3$)} & \centering{1.33333333E+000}& \centering{1.33226763E-015} & \centering{9.99200722E-016} \tabularnewline \hline
		
		\centering{(${}^5/_3$; ${}^4/_3$)} & \centering{1.66666667E+000} & \centering{6.66133815E-016} & \centering{3.99680289E-016} \tabularnewline \hline
		
		\centering{(${}^4/_3$; ${}^5/_3$)} & \centering{1.33333333E+000} & \centering{1.77635684E-015} & \centering{1.33226763E-015} \tabularnewline \hline
		
		\centering{(${}^5/_3$; ${}^5/_3$)} & \centering{1.66666667E+000} & \centering{6.66133815E-016} & \centering{3.99680289E-016} \tabularnewline \hline
		
	\end{tabularx}
	\label{tab:test2}
\end{table}

\begin{table}
	\caption{Тестирование при $u = z$, $f = \frac{z}{r^2}$, $\mu_0 = 1,\sigma = 0$}
	\centering
	\small
	\begin{tabularx}{1.0\textwidth}{| >{\raggedright\arraybackslash}X | >{\raggedright\arraybackslash}X | >{\raggedright\arraybackslash}X |>{\raggedright\arraybackslash}X |}
		\hline
		\centering{Узел} & \centering{Значение} & \centering{Абсолютная погрешность} & \centering{Относительная погрешность} \tabularnewline \hline
		
		
		
		\centering{(${}^4/_3$; ${}^4/_3$)} & \centering{1.33491362E+000}& \centering{1.58028263E-003} & \centering{1.18521198E-003} \tabularnewline \hline
		
		\centering{(${}^5/_3$; ${}^4/_3$)} & \centering{1.33426439E+000} & \centering{9.31054340E-004} & \centering{6.98290755E-004} \tabularnewline \hline
		
		\centering{(${}^4/_3$; ${}^5/_3$)} & \centering{1.66848983E+000} & \centering{1.82315862E-003} & \centering{1.09389517E-003} \tabularnewline \hline
		
		\centering{(${}^5/_3$; ${}^5/_3$)} & \centering{1.66769291E+000} & \centering{1.02624366E-003} & \centering{6.15746195E-004} \tabularnewline \hline
		
	\end{tabularx}
	\label{tab:test3}
\end{table}

\begin{table}
	\caption{Тестирование при $u = r+z$, $f = \frac{z}{r^2}$, $\mu_0 = 1,\sigma = 0$}
	\centering
	\small
	\begin{tabularx}{1.0\textwidth}{| >{\raggedright\arraybackslash}X | >{\raggedright\arraybackslash}X | >{\raggedright\arraybackslash}X |>{\raggedright\arraybackslash}X |}
		\hline
		\centering{Узел} & \centering{Значение} & \centering{Абсолютная погрешность} & \centering{Относительная погрешность} \tabularnewline \hline
		
		
		
		\centering{(${}^4/_3$; ${}^4/_3$)} & \centering{2.66824695E+000}& \centering{1.58028263E-003} & \centering{5.92605988E-004} \tabularnewline \hline
		
		\centering{(${}^5/_3$; ${}^4/_3$)} & \centering{3.00093105E+000} & \centering{9.31054340E-004} & \centering{3.10351447E-004} \tabularnewline \hline
		
		\centering{(${}^4/_3$; ${}^5/_3$)} & \centering{3.00182316E+000} & \centering{1.82315862E-003} & \centering{6.07719539E-004} \tabularnewline \hline
		
		\centering{(${}^5/_3$; ${}^5/_3$)} & \centering{3.33435958E+000} & \centering{1.02624366E-003} & \centering{3.07873097E-004} \tabularnewline \hline
		
	\end{tabularx}
	\label{tab:test4}
\end{table}

\begin{table}
	\caption{Тестирование при $u = rz$, $f = 0$, $\mu_0 = 1,\sigma = 0$}
	\centering
	\small
	\begin{tabularx}{1.0\textwidth}{| >{\raggedright\arraybackslash}X | >{\raggedright\arraybackslash}X | >{\raggedright\arraybackslash}X |>{\raggedright\arraybackslash}X |}
		\hline
		\centering{Узел} & \centering{Значение} & \centering{Абсолютная погрешность} & \centering{Относительная погрешность} \tabularnewline \hline
		
		
		
		\centering{(${}^4/_3$; ${}^4/_3$)} & \centering{1.77777778E+000}& \centering{1.11022302E-015} & \centering{6.24500451E-016} \tabularnewline \hline
		
		\centering{(${}^5/_3$; ${}^4/_3$)} & \centering{2.22222222E+000} & \centering{3.10862447E-015} & \centering{1.39888101E-015} \tabularnewline \hline
		
		\centering{(${}^4/_3$; ${}^5/_3$)} & \centering{2.22222222E+000} & \centering{8.88178420E-016} & \centering{3.99680289E-016} \tabularnewline \hline
		
		\centering{(${}^5/_3$; ${}^5/_3$)} & \centering{2.77777778E+000} & \centering{4.88498131E-015} & \centering{1.75859327E-015} \tabularnewline \hline
		
	\end{tabularx}
	\label{tab:test5}
\end{table}

\begin{table}
	\caption{Тестирование при $u = r^2 + z^2$, $f = \frac{z^2}{r^2} - 5$, $\mu_0 = 1,\sigma = 0$}
	\centering
	\small
	\begin{tabularx}{1.0\textwidth}{| >{\raggedright\arraybackslash}X | >{\raggedright\arraybackslash}X | >{\raggedright\arraybackslash}X |>{\raggedright\arraybackslash}X |}
		\hline
		\centering{Узел} & \centering{Значение} & \centering{Абсолютная погрешность} & \centering{Относительная погрешность} \tabularnewline \hline
		
		
		
		\centering{(${}^4/_3$; ${}^4/_3$)} & \centering{3.55717205E+000}& \centering{1.61649660E-003} & \centering{4.54639669E-004} \tabularnewline \hline
		
		\centering{(${}^5/_3$; ${}^4/_3$)} & \centering{4.55644336E+000} & \centering{8.87803368E-004} & \centering{1.94883666E-004} \tabularnewline \hline
		
		\centering{(${}^4/_3$; ${}^5/_3$)} & \centering{4.55790068E+000} & \centering{2.34512455E-003} & \centering{5.14783438E-004} \tabularnewline \hline
		
		\centering{(${}^5/_3$; ${}^5/_3$)} & \centering{5.55672893E+000} & \centering{1.17337132E-003} & \centering{2.11206838E-004} \tabularnewline \hline
		
	\end{tabularx}
	\label{tab:test6}
\end{table}

\begin{table}
	\caption{Тестирование при $u = r^2 z^2$, $f = -3z^2 - 2r^2$, $\mu_0 = 1,\sigma = 0$}
	\centering
	\small
	\begin{tabularx}{1.0\textwidth}{| >{\raggedright\arraybackslash}X | >{\raggedright\arraybackslash}X | >{\raggedright\arraybackslash}X |>{\raggedright\arraybackslash}X |}
		\hline
		\centering{Узел} & \centering{Значение} & \centering{Абсолютная погрешность} & \centering{Относительная погрешность} \tabularnewline \hline
		
		
		
		\centering{(${}^4/_3$; ${}^4/_3$)} & \centering{3.15919390E+000}& \centering{1.29993140E-003} & \centering{4.11306418E-004} \tabularnewline \hline
		
		\centering{(${}^5/_3$; ${}^4/_3$)} & \centering{4.93728492E+000} & \centering{9.86688136E-004} & \centering{1.99804348E-004} \tabularnewline \hline
		
		\centering{(${}^4/_3$; ${}^5/_3$)} & \centering{4.93658403E+000} & \centering{1.68757555E-003} & \centering{3.41734049E-004} \tabularnewline \hline
		
		\centering{(${}^5/_3$; ${}^5/_3$)} & \centering{7.71481536E+000} & \centering{1.23402231E-003} & \centering{1.59929291E-004} \tabularnewline \hline
		
	\end{tabularx}
	\label{tab:test7}
\end{table}

\begin{table}
	\caption{Тестирование при $u = r^3+z^3$, $f = -8r -6z + \frac{z^3}{r^2}$, $\mu_0 = 1,\sigma = 0$}
	\centering
	\small
	\begin{tabularx}{1.0\textwidth}{| >{\raggedright\arraybackslash}X | >{\raggedright\arraybackslash}X | >{\raggedright\arraybackslash}X |>{\raggedright\arraybackslash}X |}
		\hline
		\centering{Узел} & \centering{Значение} & \centering{Абсолютная погрешность} & \centering{Относительная погрешность} \tabularnewline \hline
		
		
		
		\centering{(${}^4/_3$; ${}^4/_3$)} & \centering{4.73874864E+000}& \centering{1.99210278E-003} & \centering{4.20209180E-004} \tabularnewline \hline
		
		\centering{(${}^5/_3$; ${}^4/_3$)} & \centering{6.99757994E+000} & \centering{2.42006018E-003} & \centering{3.45722883E-004} \tabularnewline \hline
		
		\centering{(${}^4/_3$; ${}^5/_3$)} & \centering{6.99968104E+000} & \centering{3.18957115E-004} & \centering{4.55653022E-005} \tabularnewline \hline
		
		\centering{(${}^5/_3$; ${}^5/_3$)} & \centering{9.25749495E+000} & \centering{1.76431155E-003} & \centering{1.90545647E-004} \tabularnewline \hline
		
	\end{tabularx}
	\label{tab:test8}
\end{table}

\begin{table}
	\caption{Тестирование при $u = r^3 z^3$, $f = -8rz^3 - 6r^3 z$, $\mu_0 = 1,\sigma = 0$}
	\centering
	\small
	\begin{tabularx}{1.0\textwidth}{| >{\raggedright\arraybackslash}X | >{\raggedright\arraybackslash}X | >{\raggedright\arraybackslash}X |>{\raggedright\arraybackslash}X |}
		\hline
		\centering{Узел} & \centering{Значение} & \centering{Абсолютная погрешность} & \centering{Относительная погрешность} \tabularnewline \hline
		
		
		
		\centering{(${}^4/_3$; ${}^4/_3$)} & \centering{5.60268110E+000}& \centering{1.59745896E-002} & \centering{2.84313374E-003} \tabularnewline \hline
		
		\centering{(${}^5/_3$; ${}^4/_3$)} & \centering{1.09603120E+001} & \centering{1.36249123E-002} & \centering{1.24157014E-003} \tabularnewline \hline
		
		\centering{(${}^4/_3$; ${}^5/_3$)} & \centering{1.09509327E+001} & \centering{2.30042146E-002} & \centering{2.09625906E-003} \tabularnewline \hline
		
		\centering{(${}^5/_3$; ${}^5/_3$)} & \centering{2.14142095E+001} & \centering{1.92609735E-002} & \centering{8.98639979E-004} \tabularnewline \hline
		
	\end{tabularx}
	\label{tab:test9}
\end{table}

Исходя из полученных данных, можно сказать, что программа верно находит численное решение задачи.

Рассмотрим решение функции $u = r^3 + z^3$, последовательно разбивая сетку в 2 раза. 

\begin{table}
	\caption{Тестирование при $u = r^3 + z^3$, $f = -8rz^3 - 6r^3 z$, $\mu_0 = 1$}
	\centering
	\small
	\begin{tabularx}{1.0\textwidth}{| >{\raggedright\arraybackslash}X | >{\raggedright\arraybackslash}X | >{\raggedright\arraybackslash}X |>{\raggedright\arraybackslash}X |}
		\hline
		\centering{Количество разбиений} & \centering{Средняя погрешность} & \centering{Порядок сходимости} \tabularnewline \hline		
		
		\centering{2} & \centering{1.7116567E-004}& \centering{-} \tabularnewline \hline
		
		\centering{4} & \centering{5.2066366E-005} & \centering{1.716969754} \tabularnewline \hline
		
		\centering{8} & \centering{1.4089198E-005} & \centering{1.885762274} \tabularnewline \hline
		
		\centering{16} & \centering{3.6602112E-006} & \centering{1.94459064} \tabularnewline \hline
		
		\centering{32} & \centering{9.3301457E-007} & \centering{1.971955381} \tabularnewline \hline
		
	\end{tabularx}
	\label{tab:test11}
\end{table}

Порядок сходимости стремится к 2.
%\input{tex/appendix3}
\chapter*{Приложение 3. Текст программы}
\addcontentsline{toc}{chapter}{Приложение 3. Текст программы}
\label{code: code}
\subsection*{Program.cs}
\lst{cs}{code/Program.cs}

\subsection*{LocalMatrix.cs}
\lst{cs}{code/LocalMatrix.cs}

\subsection*{LocalVector.cs}
\lst{cs}{code/LocalVector.cs}

\subsection*{MCG.cs}
\lst{cs}{code/MCG.cs}

\end{document}
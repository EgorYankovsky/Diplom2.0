\hypertarget{a1}{}
\chapter*{Приложение 1. Тестирование двумерной задачи на полиномах}
\addcontentsline{toc}{chapter}{Приложение 1. Тестирование двумерной задачи на полиномах}

Проведем сначала тестирование программы на работоспособность для уравнения (\ref{eq_4_1}). Образец расчетной области изображен на рисунке \ref{fig:exampleOf3DMesh}. Это область $\Omega = [1.0, 2.0]_r \times [1.0, 2.0]_z$, она содержит 16 узлов, а на всех границах будем задавать первые краевые условия.

\begin{equation} \label{eq_4_1}
	-\frac{1}{\mu_0 r} \frac{\partial}{\partial r} \left(r \frac{\partial u}{\partial r}\right) - \frac{1}{\mu_0} \frac{\partial^2 u}{\partial z^2} + \frac{u}{\mu_0 r^2} + \sigma \frac{\partial A_{\varphi}}{\partial t} = f,
\end{equation}

\begin{figure}
	\centering
	\includegraphics[width=0.75\linewidth]{images/"TestMesh".png}
	\caption{Расчетная область}
	\label{fig:exampleOf3DMesh}
\end{figure}

\begin{table}
	\caption{Тестирование при $u = 2$, $f = \frac{2}{r^2}$, $\mu_0 = 1,\sigma = 0$}
	\centering
	\small
	\begin{tabularx}{1.0\textwidth}{| >{\raggedright\arraybackslash}X | >{\raggedright\arraybackslash}X | >{\raggedright\arraybackslash}X |>{\raggedright\arraybackslash}X |}
		\hline
		\centering{Узел} & \centering{Значение} & \centering{Абсолютная погрешность} & \centering{Относительная погрешность} \tabularnewline \hline
		
		
		
		\centering{(${}^4/_3$; ${}^4/_3$)} & \centering{2.00226896E+000}& \centering{2.26896083E-003} & \centering{1.13448042E-003} \tabularnewline \hline
		
		\centering{(${}^5/_3$; ${}^4/_3$)} & \centering{2.00130487E+000} & \centering{1.30486533E-003} & \centering{6.52432666E-004} \tabularnewline \hline
		
		\centering{(${}^4/_3$; ${}^5/_3$)} & \centering{2.00226896E+000} & \centering{2.26896083E-003} & \centering{1.13448042E-003} \tabularnewline \hline
		
		\centering{(${}^5/_3$; ${}^5/_3$)} & \centering{2.00130487E+000} & \centering{1.30486533E-003} & \centering{6.52432666E-004} \tabularnewline \hline
		
	\end{tabularx}
	\label{tab:test1}
\end{table}

\begin{table}
	\caption{Тестирование при $u = r$, $f = 0$, $\mu_0 = 1,\sigma = 0$}
	\centering
	\small
	\begin{tabularx}{1.0\textwidth}{| >{\raggedright\arraybackslash}X | >{\raggedright\arraybackslash}X | >{\raggedright\arraybackslash}X |>{\raggedright\arraybackslash}X |}
		\hline
		\centering{Узел} & \centering{Значение} & \centering{Абсолютная погрешность} & \centering{Относительная погрешность} \tabularnewline \hline
		
		
		
		\centering{(${}^4/_3$; ${}^4/_3$)} & \centering{1.33333333E+000}& \centering{1.33226763E-015} & \centering{9.99200722E-016} \tabularnewline \hline
		
		\centering{(${}^5/_3$; ${}^4/_3$)} & \centering{1.66666667E+000} & \centering{6.66133815E-016} & \centering{3.99680289E-016} \tabularnewline \hline
		
		\centering{(${}^4/_3$; ${}^5/_3$)} & \centering{1.33333333E+000} & \centering{1.77635684E-015} & \centering{1.33226763E-015} \tabularnewline \hline
		
		\centering{(${}^5/_3$; ${}^5/_3$)} & \centering{1.66666667E+000} & \centering{6.66133815E-016} & \centering{3.99680289E-016} \tabularnewline \hline
		
	\end{tabularx}
	\label{tab:test2}
\end{table}

\begin{table}
	\caption{Тестирование при $u = z$, $f = \frac{z}{r^2}$, $\mu_0 = 1,\sigma = 0$}
	\centering
	\small
	\begin{tabularx}{1.0\textwidth}{| >{\raggedright\arraybackslash}X | >{\raggedright\arraybackslash}X | >{\raggedright\arraybackslash}X |>{\raggedright\arraybackslash}X |}
		\hline
		\centering{Узел} & \centering{Значение} & \centering{Абсолютная погрешность} & \centering{Относительная погрешность} \tabularnewline \hline
		
		
		
		\centering{(${}^4/_3$; ${}^4/_3$)} & \centering{1.33491362E+000}& \centering{1.58028263E-003} & \centering{1.18521198E-003} \tabularnewline \hline
		
		\centering{(${}^5/_3$; ${}^4/_3$)} & \centering{1.33426439E+000} & \centering{9.31054340E-004} & \centering{6.98290755E-004} \tabularnewline \hline
		
		\centering{(${}^4/_3$; ${}^5/_3$)} & \centering{1.66848983E+000} & \centering{1.82315862E-003} & \centering{1.09389517E-003} \tabularnewline \hline
		
		\centering{(${}^5/_3$; ${}^5/_3$)} & \centering{1.66769291E+000} & \centering{1.02624366E-003} & \centering{6.15746195E-004} \tabularnewline \hline
		
	\end{tabularx}
	\label{tab:test3}
\end{table}

\begin{table}
	\caption{Тестирование при $u = r+z$, $f = \frac{z}{r^2}$, $\mu_0 = 1,\sigma = 0$}
	\centering
	\small
	\begin{tabularx}{1.0\textwidth}{| >{\raggedright\arraybackslash}X | >{\raggedright\arraybackslash}X | >{\raggedright\arraybackslash}X |>{\raggedright\arraybackslash}X |}
		\hline
		\centering{Узел} & \centering{Значение} & \centering{Абсолютная погрешность} & \centering{Относительная погрешность} \tabularnewline \hline
		
		
		
		\centering{(${}^4/_3$; ${}^4/_3$)} & \centering{2.66824695E+000}& \centering{1.58028263E-003} & \centering{5.92605988E-004} \tabularnewline \hline
		
		\centering{(${}^5/_3$; ${}^4/_3$)} & \centering{3.00093105E+000} & \centering{9.31054340E-004} & \centering{3.10351447E-004} \tabularnewline \hline
		
		\centering{(${}^4/_3$; ${}^5/_3$)} & \centering{3.00182316E+000} & \centering{1.82315862E-003} & \centering{6.07719539E-004} \tabularnewline \hline
		
		\centering{(${}^5/_3$; ${}^5/_3$)} & \centering{3.33435958E+000} & \centering{1.02624366E-003} & \centering{3.07873097E-004} \tabularnewline \hline
		
	\end{tabularx}
	\label{tab:test4}
\end{table}

\begin{table}
	\caption{Тестирование при $u = rz$, $f = 0$, $\mu_0 = 1,\sigma = 0$}
	\centering
	\small
	\begin{tabularx}{1.0\textwidth}{| >{\raggedright\arraybackslash}X | >{\raggedright\arraybackslash}X | >{\raggedright\arraybackslash}X |>{\raggedright\arraybackslash}X |}
		\hline
		\centering{Узел} & \centering{Значение} & \centering{Абсолютная погрешность} & \centering{Относительная погрешность} \tabularnewline \hline
		
		
		
		\centering{(${}^4/_3$; ${}^4/_3$)} & \centering{1.77777778E+000}& \centering{1.11022302E-015} & \centering{6.24500451E-016} \tabularnewline \hline
		
		\centering{(${}^5/_3$; ${}^4/_3$)} & \centering{2.22222222E+000} & \centering{3.10862447E-015} & \centering{1.39888101E-015} \tabularnewline \hline
		
		\centering{(${}^4/_3$; ${}^5/_3$)} & \centering{2.22222222E+000} & \centering{8.88178420E-016} & \centering{3.99680289E-016} \tabularnewline \hline
		
		\centering{(${}^5/_3$; ${}^5/_3$)} & \centering{2.77777778E+000} & \centering{4.88498131E-015} & \centering{1.75859327E-015} \tabularnewline \hline
		
	\end{tabularx}
	\label{tab:test5}
\end{table}

\begin{table}
	\caption{Тестирование при $u = r^2 + z^2$, $f = \frac{z^2}{r^2} - 5$, $\mu_0 = 1,\sigma = 0$}
	\centering
	\small
	\begin{tabularx}{1.0\textwidth}{| >{\raggedright\arraybackslash}X | >{\raggedright\arraybackslash}X | >{\raggedright\arraybackslash}X |>{\raggedright\arraybackslash}X |}
		\hline
		\centering{Узел} & \centering{Значение} & \centering{Абсолютная погрешность} & \centering{Относительная погрешность} \tabularnewline \hline
		
		
		
		\centering{(${}^4/_3$; ${}^4/_3$)} & \centering{3.55717205E+000}& \centering{1.61649660E-003} & \centering{4.54639669E-004} \tabularnewline \hline
		
		\centering{(${}^5/_3$; ${}^4/_3$)} & \centering{4.55644336E+000} & \centering{8.87803368E-004} & \centering{1.94883666E-004} \tabularnewline \hline
		
		\centering{(${}^4/_3$; ${}^5/_3$)} & \centering{4.55790068E+000} & \centering{2.34512455E-003} & \centering{5.14783438E-004} \tabularnewline \hline
		
		\centering{(${}^5/_3$; ${}^5/_3$)} & \centering{5.55672893E+000} & \centering{1.17337132E-003} & \centering{2.11206838E-004} \tabularnewline \hline
		
	\end{tabularx}
	\label{tab:test6}
\end{table}

\begin{table}
	\caption{Тестирование при $u = r^2 z^2$, $f = -3z^2 - 2r^2$, $\mu_0 = 1,\sigma = 0$}
	\centering
	\small
	\begin{tabularx}{1.0\textwidth}{| >{\raggedright\arraybackslash}X | >{\raggedright\arraybackslash}X | >{\raggedright\arraybackslash}X |>{\raggedright\arraybackslash}X |}
		\hline
		\centering{Узел} & \centering{Значение} & \centering{Абсолютная погрешность} & \centering{Относительная погрешность} \tabularnewline \hline
		
		
		
		\centering{(${}^4/_3$; ${}^4/_3$)} & \centering{3.15919390E+000}& \centering{1.29993140E-003} & \centering{4.11306418E-004} \tabularnewline \hline
		
		\centering{(${}^5/_3$; ${}^4/_3$)} & \centering{4.93728492E+000} & \centering{9.86688136E-004} & \centering{1.99804348E-004} \tabularnewline \hline
		
		\centering{(${}^4/_3$; ${}^5/_3$)} & \centering{4.93658403E+000} & \centering{1.68757555E-003} & \centering{3.41734049E-004} \tabularnewline \hline
		
		\centering{(${}^5/_3$; ${}^5/_3$)} & \centering{7.71481536E+000} & \centering{1.23402231E-003} & \centering{1.59929291E-004} \tabularnewline \hline
		
	\end{tabularx}
	\label{tab:test7}
\end{table}

\begin{table}
	\caption{Тестирование при $u = r^3+z^3$, $f = -8r -6z + \frac{z^3}{r^2}$, $\mu_0 = 1,\sigma = 0$}
	\centering
	\small
	\begin{tabularx}{1.0\textwidth}{| >{\raggedright\arraybackslash}X | >{\raggedright\arraybackslash}X | >{\raggedright\arraybackslash}X |>{\raggedright\arraybackslash}X |}
		\hline
		\centering{Узел} & \centering{Значение} & \centering{Абсолютная погрешность} & \centering{Относительная погрешность} \tabularnewline \hline
		
		
		
		\centering{(${}^4/_3$; ${}^4/_3$)} & \centering{4.73874864E+000}& \centering{1.99210278E-003} & \centering{4.20209180E-004} \tabularnewline \hline
		
		\centering{(${}^5/_3$; ${}^4/_3$)} & \centering{6.99757994E+000} & \centering{2.42006018E-003} & \centering{3.45722883E-004} \tabularnewline \hline
		
		\centering{(${}^4/_3$; ${}^5/_3$)} & \centering{6.99968104E+000} & \centering{3.18957115E-004} & \centering{4.55653022E-005} \tabularnewline \hline
		
		\centering{(${}^5/_3$; ${}^5/_3$)} & \centering{9.25749495E+000} & \centering{1.76431155E-003} & \centering{1.90545647E-004} \tabularnewline \hline
		
	\end{tabularx}
	\label{tab:test8}
\end{table}

\begin{table}
	\caption{Тестирование при $u = r^3 z^3$, $f = -8rz^3 - 6r^3 z$, $\mu_0 = 1,\sigma = 0$}
	\centering
	\small
	\begin{tabularx}{1.0\textwidth}{| >{\raggedright\arraybackslash}X | >{\raggedright\arraybackslash}X | >{\raggedright\arraybackslash}X |>{\raggedright\arraybackslash}X |}
		\hline
		\centering{Узел} & \centering{Значение} & \centering{Абсолютная погрешность} & \centering{Относительная погрешность} \tabularnewline \hline
		
		
		
		\centering{(${}^4/_3$; ${}^4/_3$)} & \centering{5.60268110E+000}& \centering{1.59745896E-002} & \centering{2.84313374E-003} \tabularnewline \hline
		
		\centering{(${}^5/_3$; ${}^4/_3$)} & \centering{1.09603120E+001} & \centering{1.36249123E-002} & \centering{1.24157014E-003} \tabularnewline \hline
		
		\centering{(${}^4/_3$; ${}^5/_3$)} & \centering{1.09509327E+001} & \centering{2.30042146E-002} & \centering{2.09625906E-003} \tabularnewline \hline
		
		\centering{(${}^5/_3$; ${}^5/_3$)} & \centering{2.14142095E+001} & \centering{1.92609735E-002} & \centering{8.98639979E-004} \tabularnewline \hline
		
	\end{tabularx}
	\label{tab:test9}
\end{table}

Исходя из полученных данных, можно сказать, что программа верно находит численное решение задачи.

Рассмотрим решение функции $u = r^3 + z^3$, последовательно разбивая сетку в 2 раза. 

\begin{table}
	\caption{Тестирование при $u = r^3 + z^3$, $f = -8rz^3 - 6r^3 z$, $\mu_0 = 1$}
	\centering
	\small
	\begin{tabularx}{1.0\textwidth}{| >{\raggedright\arraybackslash}X | >{\raggedright\arraybackslash}X | >{\raggedright\arraybackslash}X |>{\raggedright\arraybackslash}X |}
		\hline
		\centering{Количество разбиений} & \centering{Средняя погрешность} & \centering{Порядок сходимости} \tabularnewline \hline		
		
		\centering{2} & \centering{1.7116567E-004}& \centering{-} \tabularnewline \hline
		
		\centering{4} & \centering{5.2066366E-005} & \centering{1.716969754} \tabularnewline \hline
		
		\centering{8} & \centering{1.4089198E-005} & \centering{1.885762274} \tabularnewline \hline
		
		\centering{16} & \centering{3.6602112E-006} & \centering{1.94459064} \tabularnewline \hline
		
		\centering{32} & \centering{9.3301457E-007} & \centering{1.971955381} \tabularnewline \hline
		
	\end{tabularx}
	\label{tab:test11}
\end{table}

Порядок сходимости стремится к 2.
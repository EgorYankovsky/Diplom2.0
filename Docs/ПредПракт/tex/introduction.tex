\chapter*{Введение}

\addcontentsline{toc}{chapter}{Введение}

С увеличением потребности в природных ресурсах развивались способы поиска и исследования земных пород и руд. Наиболее распространёнными являются гравиразведка, магниторазведка, и электроразведка. Первые два можно считать естественными или природными, поскольку в их основе лежит использование гравитационного и магнитного полей Земли. При использовании электроразведки, поле создаётся искусственными источниками. В качестве данных источников можно рассматривать петлевые источники, вертикальные электрические линии (ВЭЛ) и самый новый из предложенных -- круговой электрический диполь (КЭД). Новизна его заключается в том, что это источник переменного поля, наземный аналог вертикальной электрической линии. 

Помимо возможности нахождения параметров среды, что является обратной задачей по определению, можно изучать и поведение самого электромагнитного поля. Зная его источник и параметры среды можно воспроизвести поле в пространстве земной коры, изучать характер его поведения в зависимости от количества неоднородностей в земной среде или параметров горизонтально-слоистой среды Земли.

При моделирования таких процессов, а особенно при истолковании сложных полей в которых было необходимо преобразовывать таким образом, чтобы разделить аномалии в зависимости от глубины расположения источников поля и обособлять такие изменения поля, которые соответствуют аномалиям тел простейших форм, обычные аналитические методы математического моделирования не помогут в силу больших временных затрат на расчёты. Вместо этого пользуются различными численными методами на различных ЭВМ. Для моделирования электромагнитных полей, лучшим вариантом можно считать метод конечных элементов (МКЭ). 

В данной работе будет рассмотренна возможность расчёта нестационарного поля КЭД в осесимметричной среде. Исследования проводились на воображаемой области размером $\Omega \in [0.001; 1000]_r \times [-1000; 0]_z$. Данная область включает в себя несколько разных слоев земных пород, имеющих различные удельные значения физических величин. Также рассмотрим процесс добавления нескольких аномальных пород, характерных различным земным рудам.

При написании работы использовались следующие языки программирования: для математических расчётов -- C$\#$ 12 на платформе .NET 8.0, для визуализации полученных результатов -- Python 3.12.2 с пакетами matplotlib версии 3.8.2 и numpy версии 1.26.4. 

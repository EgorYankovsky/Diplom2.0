\chapter{Практическая часть}

\section{Формат входных и выходных данных}

Входные данные содержатся в папке "Data/Input/". Файл "WholeMesh.txt" содержит данные о трёхмерной сетке, из которой автоматически строится сетка для решения двумерной задачи на нормальном поле. В файле содержится информация о границах расчётной  области по $x$, $y$, $z$, количество необходимых разбиений для каждой оси, коэффициенты разрядки, количество областей с разными значениями удельной электропроводности и информация о границах расчётной области. Полностью формат изображен на рисунке \ref{fig:TextWholeMesh}.

\begin{figure}
	\centering
	\includegraphics[width=0.8\linewidth]{images/"inputMeshText".png}
	\caption{Входной формат сетки по пространству}
	\label{fig:TextWholeMesh}
\end{figure}

Входные данные для учёта поля влияния хранятся в папке "Data/Input/Anomalies/". Каждый файл, находящийся в этой папке, содержит примерно похожий формат хранения, как и для основной сетки. Задаются границы по осям $x$, $y$, $z$, количество необходимых разбиений для каждой оси, коэффициенты разрядки, значения удельной электропроводности на аномальной области и границы этой области. Полностью формат изображен на рисунке \ref{fig:TextAnomalyMesh}.

\begin{figure}
	\centering
	\includegraphics[width=0.8\linewidth]{images/"inputAnomalyText".png}
	\caption{Входной формат сетки по пространству для аномалии}
	\label{fig:TextAnomalyMesh}
\end{figure}

Входные данные для сетки по времени, содержатся в файле "Time.txt", в папке "Data/Input/" и содержат четыре значения: время начала и конца, количество разбиений и коэффициент разрядки.

\section{Сборка глобальной матрицы и глобального вектора правой части}

При формировании матрицы \textbf{A} для решения СЛАУ необходимо учитывать соответствие локальной к глобальной нумерации каждого узла. Глобальная нумерация узлов сетки однозначно определяет вклад локальной матрицы в соответствующие строчки и столбцы матрицы \textbf{A}. Поэтому, зная глобальную нумерацию узлов конечного элемента, можно определить какие элементы глобальной матрицы изменятся при добавлении в нее локальной. Аналогичным образом определяется вклад локального вектора правой части в глобальный.

\lst{cs}{code/AddLocalMatrix.cs}

\section{Учёт краевых условий}

Поскольку в решаемой задаче у нас на всех границах задаётся однородное краевое условие первого рода, технически необходимо в соответствующей строчке матрицы обнулить вне диагональные элементы, на диагонали поставить значение 1, а в соответствующую строчку вектора правой части поставить значение краевого условия на этой границе, т.е. в нашем случае тоже обнулить.

\section{Решение СЛАУ}

Для решения СЛАУ мы будем использовать локально-оптимальную схему \textbf{(очень хочется починить LU предобусловливание)}. Это хороший метод решения систем уравнений для несимметричных матриц. Перед решением СЛАУ задаются параметры для досрочного выхода из итерационного процесса, а именно: выход по максимальному количеству совершённых итераций и минимальному значению нормы вектора невязки.

\lst{cs}{code/LOS.cs}

В результате решения СЛАУ мы получим вектор $q$ весов базисных функций, на которые раскладывается функция $A_{\varphi}^0$ или вектор-функция $\overrightarrow{\textbf{A}}^{+}$. Учитывая построение базисных функций, компонентами этого вектора будут значения функции в соответствующих узлах сетки.

\section{Определение значения вектор-потенциала и напряжённости электрического поля}

После решения СЛАУ вида (\ref{eq_2_27}) или (\ref{eq_2_43}), необходимо найти напряжённость электрического поля по формуле (\ref{eq_2_28}). Пользуясь аналитическим представлением из (\ref{eq_2_23}), программная реализация будет выглядеть следующим образом.

\lst{cs}{code/E_generator.cs}

Поскольку в качестве конечных элементов использовались прямоугольники для двумерной и прямые параллелепипеды для трёхмерной задач, то можно упростить алгоритм нахождения значения функции на элементе. Можно не перебирать каждый элемент отдельно и проверять значение интересующей точки на принадлежность ему, а последовательно сравнивать координаты точки со значениями на разбиениях по осям координат. Тогда сложность алгоритма будет не $O(n^2)$ для двумерной или $O(n^3)$ для трёхмерной задач, а $O(k \cdot n)$, где $n$ -- количество отрезков, на которые разбиваются оси координат.

\lst{cs}{code/Checker.cs}
 
\section{Проверка полученных результатов}

Проверку полученных результатов решения СЛАУ будем из закона индукции Фарадея (\ref{eq_1_2}) и теоремы о циркуляции магнитного поля (\ref{eq_1_1}). Учитывая (\ref{eq_2_27}) -- (\ref{eq_2_29}), получим выражение для $\overrightarrow{\textbf{B}}$:

\begin{equation} \label{eq_3_1}
	\overrightarrow{\textbf{B}} = \text{rot} \overrightarrow{\textbf{A}} = 
	\begin{vmatrix}
		\textbf{i} & \textbf{j} & \textbf{k}\\
		\frac{\partial}{\partial x} & \frac{\partial}{\partial y} & \frac{\partial}{\partial z}\\
		A_x & A_y & 0
	\end{vmatrix}
	= -\frac{\partial A_y}{\partial z} \textbf{i} + \frac{\partial A_x}{\partial z} \textbf{j} + \left(\frac{\partial A_y}{\partial x} - \frac{\partial A_x}{\partial y}\right) \textbf{k}.
\end{equation}

Исходя из теории конечно-разностных схем \cite{7}, численно определим значения для частных первых производных в выражении (\ref{eq_3_1}):

\begin{equation} \label{eq_3_2}
	\frac{\partial A_{x_i}}{\partial x_k} + o\left(h_{x_k}^3\right) = \frac{A_{x_i}^{j+1} - A_{x_i}^{j-1}}{2h_{x_k}},
\end{equation}
где $x_k$ -- переменная по которой проводится дифференцирование, $x_i$ -- соответствующая компонента вектора-потенциала $\overrightarrow{\textbf{A}}$, $A_{x_i}^{j+1} = A_{x_i}\left(x_{0i} + h_{x_k}\right)$, $A_{x_i}^{j-1} = A_{x_{i}}\left(x_{0i} - h_{x_k}\right)$, $h_{x_k}$ -- шаг от точки, в которой необходимо найти значение производной, равный $10^{-10}$.

\textbf{Здесь будет программная реализация $\downarrow$}


\chapter{Постановка задачи}

\section{Аппарат математического моделирования}

Математическая модель, описывающая поведение электромагнитного поля в пространстве, известна в наши дни, как система уравнений Максвелла. Она позволяет описывать взаимосвязь сразу нескольких физических величин: напряжённости электрического $\overrightarrow{\textbf{E}}$ и магнитного $\overrightarrow{\textbf{H}}$ полей, а также индукцию магнитного поля $\overrightarrow{\textbf{B}}$. Большинство вычислительных задач электромагнетизма базируются на дифференциальной форме системы уравнений Максвелла (\ref{eq_1_1}) -- (\ref{eq_1_4}):

\begin{equation} \label{eq_1_1}
	\text{rot} \overrightarrow{\textbf{H}} = \overrightarrow{\textbf{J}^{\text{ст}}} + \sigma \overrightarrow{\textbf{E}} + \frac{\partial \left(\varepsilon \overrightarrow{\textbf{E}} \right)}{\partial t},
\end{equation}

\begin{equation} \label{eq_1_2}
	\text{rot} \overrightarrow{\textbf{E}} = - \frac{\partial \overrightarrow{\textbf{B}}}{\partial t},
\end{equation}

\begin{equation} \label{eq_1_3}
	\text{div} \overrightarrow{\textbf{B}} = 0,
\end{equation}

\begin{equation} \label{eq_1_4}
	\text{div} \varepsilon \overrightarrow{\textbf{E}} = \rho,
\end{equation}
где $\overrightarrow{\textbf{J}^{\text{ст}}}$ -- вектор плотностей сторонних токов, $\sigma$ -- удельная электрическая проводимость среды, $\varepsilon$ -- диэлектрическая проницаемость среды, а $\rho$ -- объёмная плотность стороннего электрического заряда.

Основное преимущество использования системы уравнений (\ref{eq_1_1}) -- (\ref{eq_1_4}) в дифференциальной форме, заключается в возможности учитывать нелинейность, анизотропию и другие нетривиальные аспекты среды расчётной области \cite{2}. 

Предлагаемый в работе подход к математическому конечноэлементному моделированию основан на технологии разделения полей, позволяющей существенно сократить вычислительные затраты. В рассматриваемой задаче под неоднородностями (аномалиями) будем понимать трёхмерные геологические объекты, отличные от сопротивления вмещающей горизонтально-слоистой среды.

Будем считать, что электромагнитное поле возбуждается круговым источником тока. В таком случае, в силу симметрии расчётной области будем решать задачу в цилиндрических координатах. Источник поля в таком случае описывается точкой, расположенной на некотором расстоянии, достаточно далёком от границы расчётной области. Тогда при условии однородности среды по магнитной проницаемости электромагнитное поле полностью описывается одной компонентой $A_{\varphi} = A_{\varphi}(r, z, t)$ вектор-потенциала $\overrightarrow{\textbf{A}}$. Функция $A_{\varphi}(r, z, t)$ может быть найдена из решения двумерного уравнения (\ref{eq_1_5}):

\begin{equation} \label{eq_1_5}
	-\frac{1}{\mu_0} \Delta A_{\varphi} + \frac{A_{\varphi}}{\mu_0 r^2} + \sigma \frac{\partial A_{\varphi}}{\partial t} = J_{\varphi},
\end{equation}
где $J_{\varphi}$ - источник стороннего тока, описываемый дельта-функцией, равной 1 в одной из подобласти, описывающей источник поля, и 0 во всех остальных. Удельную электропроводность $\sigma$ представим в виде кусочно-постоянной функции, описывающей физические характеристики горизонтально-слоистой среды. Потребуем, чтобы на всех границах было главное краевое условие $\left.A_{\varphi}(r, z, t)\right|_s = 0$. Тогда решение задачи (\ref{eq_1_5}) с главными однородными условиями на границах будем называть первичным или нормальным полем.

Решением задачи на оценку влияния аномальных объектов в горизонтально-слоистой среде будем называть вторичным (добавочным) полем. Также, как и в (\ref{eq_1_5}) потребуем на всех границах главное однородное краевое условие $\overrightarrow{\textbf{A}} \times \overrightarrow{\textbf{n}} |_s = 0$. Тогда, нестационарный процесс, возникающий после выключения источника тока в круглой обмотке, описывается следствием из уравнения (\ref{eq_1_1}):

\begin{equation} \label{eq_1_6}
	\text{rot} \left( \frac{1}{\mu_o} \text{rot} \overrightarrow{\textbf{A}}^{+} \right) + \sigma \frac{\partial \overrightarrow{\textbf{A}}^{+}}{\partial t} = (\sigma - \sigma_n) \overrightarrow{\textbf{E}}^n,
\end{equation}
где $\mu_0 = 4 \cdot \pi \cdot 10^{-7} = 1.25663753 \cdot 10^{-6}$ Гн/м, $\sigma_n$ -- значение удельной электрической проводимости среды на нормальном слое, $\overrightarrow{\textbf{E}}^n$ -- напряжённость первичного электрического поля, $\overrightarrow{\textbf{A}}^{+}$ -- значение вектор-потенциала на добавочном поле.

\section{Описание расчётной области}

Пусть у нас имеется расчётная область, геометрически представленная в виде параллелепипеда: $\Omega \in [-5500, 5500]_x \times [-5500, 5500]_y \times [-1500, 1500]_z$. Внутри неё имеются слои воздуха, глинозёма, песчаника, известняка и слюдяного сланца. Принимая во внимание факт того, что удельное электрическое сопротивление обратно пропорционально значению удельной электропроводности материала \cite{3}, тогда, зная значения сопротивления \cite{4}, найдём электропроводность интересующих материалов. Поскольку воздух является диэлектриком, значение удельной электропроводности для него $\sigma_{\text{возд.}} = 0$ См/м. Значения остальных слоев и аномальных областей \cite{5} приведены в таблице \ref{tab:fields}.

\begin{table}
	\caption{Удельные физические величины некоторых типов земных сред.}
	\centering
	\small
	\begin{tabularx}{1.0\textwidth}{| >{\raggedright\arraybackslash}X | >{\raggedright\arraybackslash}X | >{\raggedright\arraybackslash}X |}
		\hline
		\centering{Земная порода} & \centering{Удельное электрическое сопротивление Ом $\cdot$ м} & \centering{Удельная электропроводимость См / м}  \tabularnewline \hline
		
		\centering{Глинозём} & \centering{125}& \centering{0.008} \tabularnewline \hline
		
		\centering{Песчаник} & \centering{2000} & \centering{0.0005} \tabularnewline \hline
		
		\centering{Юрский известняк} & \centering{40} & \centering{0.025} \tabularnewline \hline
		
		\centering{Слюдяной сланец} & \centering{800} & \centering{0.00125} \tabularnewline \hline
		
		\centering{Осадочная порода} & \centering{0.1} &\centering{10} \tabularnewline \hline
		
		\centering{Грунтовые воды} & \centering{0.25} & \centering{4} \tabularnewline \hline
		
	\end{tabularx}
	\label{tab:fields}
\end{table}

Для решения задачи в максимально упрощённой, горизонтально-слоистой среде, зададим следующие значения где воздух находится на высоте $h_z = [0; 200]$, глинозём на глубине $h_z = [-9; 0]_z$, песчаник на $h_z = [-50; -9]$, известняк на $h_z = [-125; -50]$ и слюдяной сланец на $h_z = [-200; -125]$. В таком случае сама область в её двумерном случае изображена на рисунке \ref{fig:example}.

\begin{figure}
	\centering
	\includegraphics[width=1.0\linewidth]{images/"Figure_example".png}
	\caption{Расчётная область для двумерной задачи.}
	\label{fig:example}
\end{figure}
\chapter*{Заключение}

\addcontentsline{toc}{chapter}{Заключение}

В выпускной квалификационной работе была разработанная программа для расчёта электромагнитного поля в трёхмерном пространстве. 

Для проверки корректности работы программы была проведена ее верификация на полиномиальных функциях и вектор-функциях. В процессе тестирования осесимметричной задачи было получено, что на полиномах первой степени задача решается без погрешности, однако начиная с полинома второй степени появлялась погрешность, которая уменьшалась при дроблении сетки. Был рассчитан порядок сходимости метода решения, который, как и предполагалось, оказался равен порядку сходимости билинейных базисных функций. В процессе тестирования трёхмерных задач векторным методом конечных элементов результат оказался аналогичный результату осесимметричной задачи. 

Было проведено исследование на поведение электромагнитного поля, при добавлении аномалий в разные места горизонтально-слоистой среды многоэтапной схемой разделения полей. По итогам исследования была проведена оценка поведения поля при различном использовании схемы разделения. Порядок добавления аномалий в область не дал никакого влияния, т.е. порядок добавления объектов не имеет разницы при разделении полей. Также было выяснено, что при достаточно близком расположении аномальных объектов друг к другу может возникать явление взаимоиндукции двух тел. Соответственно, при использовании многоэтапной схемы разделения полей не рекомендуется пренебрегать учётом влияния других аномальных тел, расположенных на достаточно близком друг к другу расстоянии.

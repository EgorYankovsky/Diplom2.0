\chapter{Постановка задачи}

\section{Аппарат математического моделирования}

Математическая модель, описывающая поведение электромагнитного поля в пространстве, известна в наши дни, как система уравнений Максвелла. Она позволяет описывать взаимосвязь сразу нескольких физических величин: напряжённости электрического $\overrightarrow{\textbf{E}}$ и магнитного $\overrightarrow{\textbf{H}}$ полей, а также индукцию магнитного поля $\overrightarrow{\textbf{B}}$. Большинство вычислительных задач электромагнетизма базируются на дифференциальной форме системы уравнений Максвелла:

\begin{equation} \label{eq_1_1}
	\text{rot} \overrightarrow{\textbf{H}} = \overrightarrow{\textbf{J}^{\text{ст}}} + \sigma \overrightarrow{\textbf{E}} + \frac{\partial \left(\varepsilon \overrightarrow{\textbf{E}} \right)}{\partial t},
\end{equation}

\begin{equation} \label{eq_1_2}
	\text{rot} \overrightarrow{\textbf{E}} = - \frac{\partial \overrightarrow{\textbf{B}}}{\partial t},
\end{equation}

\begin{equation} \label{eq_1_3}
	\text{div} \overrightarrow{\textbf{B}} = 0,
\end{equation}

\begin{equation} \label{eq_1_4}
	\text{div} \varepsilon \overrightarrow{\textbf{E}} = \rho,
\end{equation}
где $\overrightarrow{\textbf{J}^{\text{ст}}}$ -- вектор плотностей сторонних токов, $\sigma$ -- удельная электрическая проводимость среды, $\varepsilon$ -- диэлектрическая проницаемость среды, а $\rho$ -- объёмная плотность стороннего электрического заряда.

Основное преимущество использования системы уравнений (\ref{eq_1_1}) -- (\ref{eq_1_4}) в дифференциальной форме, заключается в возможности учитывать нелинейность, анизотропию и другие нетривиальные аспекты среды \cite{3}. 

Пусть электромагнитное поле возбуждается индукционным источником. В таком случае, при отсутствии аномальных объектов, будем решать задачу в цилиндрических координатах. Источник поля в таком случае описывается точкой, расположенной на некотором расстоянии, достаточно далёком от границы расчётной области. Тогда при условии однородности среды по магнитной проницаемости и отстутствия токов смещения электромагнитное поле полностью описывается одной компонентой $A_{\varphi} = A_{\varphi}(r, z, t)$ вектор-потенциала $\overrightarrow{\textbf{A}}$. Функция $A_{\varphi}(r, z, t)$ может быть найдена из решения двумерного уравнения (\ref{eq_1_5}):

\begin{equation} \label{eq_1_5}
	-\frac{1}{\mu_0} \Delta A_{\varphi} + \frac{A_{\varphi}}{\mu_0 r^2} + \sigma \frac{\partial A_{\varphi}}{\partial t} = J_{\varphi},
\end{equation}
где $\mu_0 = 4 \cdot \pi \cdot 10^{-7} = 1.25663753 \cdot 10^{-6}$ Гн/м -- магнитная постоянная, $J_{\varphi}$ - источник стороннего тока, описываемый дельта-функцией, равной 1 в одной из подобласти, описывающей источник поля, и 0 во всех остальных \cite{4}. Удельную электропроводность $\sigma$ представим в виде кусочно-постоянной функции, описывающей физические характеристики горизонтально-слоистой среды. Потребуем, чтобы на всех границах было главное краевое условие $\left.A_{\varphi}(r, z, t)\right|_s = 0$. Тогда решение задачи (\ref{eq_1_5}) с главными однородными условиями на границах будем называть первичным или нормальным полем.

Решением задачи на оценку влияния аномальных объектов в горизонтально-слоистой среде будем называть вторичным (добавочным) полем. Также, как и в (\ref{eq_1_5}) потребуем на всех границах главное однородное краевое условие $\overrightarrow{\textbf{A}} \times \overrightarrow{\textbf{n}} |_s = 0$. Тогда, нестационарный процесс, возникающий после выключения источника тока в круглой обмотке, описывается следствием из уравнения (\ref{eq_1_1}) \cite{5}:

\begin{equation} \label{eq_1_6}
	\text{rot} \left( \frac{1}{\mu_o} \text{rot} \overrightarrow{\textbf{A}}^{+} \right) + \sigma \frac{\partial \overrightarrow{\textbf{A}}^{+}}{\partial t} = (\sigma - \sigma_n) \overrightarrow{\textbf{E}}^n,
\end{equation}
где $\sigma_n$ -- значение удельной электрической проводимости среды на нормальном слое, $\overrightarrow{\textbf{E}}^n$ -- напряжённость первичного электрического поля, $\overrightarrow{\textbf{A}}^{+}$ -- значение вектор-потенциала на добавочном поле.

\section{Описание расчётной области}

Пусть у нас имеется расчётная область, геометрически представленная в виде параллелепипеда: $\Omega \in [-55000, 55000]_x \times [-55000, 55000]_y \times [-25000, 25000]_z$. Внутри неё имеются слои воздуха, и некоторых пород верхних слоёв земной коры. Тогда половина продольного диагонального среза горизонтально-слоистой среды изображена на рисунке \ref{fig:example}. Будем её использовать в качестве расчётной области для двумерной задачи. В среде, обозначенной коричневым цветом задано значение $\sigma_1 = 0.01$ См/м, в бледной $\sigma_2 = 0.005$ См/м и в зелёной $\sigma_3 = 0.001$ См/м. Поскольку воздух является диэлектриком, значение удельной электропроводности для него $\sigma_{\text{возд.}} = 0$ См/м.

\begin{figure}
	\centering
	\vspace*{0.7cm}
	\includegraphics[width=1.0\linewidth]{images/"Figure_example".png}
	\caption{Срез горизонтально-слоистой среды}
	\label{fig:example}
\end{figure}